%%%----Documentstyles-----------------------
\documentclass[12pt,a4paper]{amsart}
\setlength{\textwidth}{\paperwidth}\addtolength{\textwidth}{-2in}\calclayout
\usepackage{hyperref}
\usepackage{comment}
\usepackage{mathtools}
\DeclarePairedDelimiter\ceil{\lceil}{\rceil}
\DeclarePairedDelimiter\floor{\lfloor}{\rfloor}

%%%----Environments------------------------
\newtheorem{thm}{Theorem}[section]
\newtheorem{lem}[thm]{Lemma}
\newtheorem{prop}[thm]{Proposition}
\numberwithin{equation}{section}

%%%---------Top matter----------------
\title[Responses]{\large Responses to the referees' comments}
%%%----Main-------------------------------
\begin{document}
\maketitle	
	We are grateful to the referees for helpful comments on the paper.
	We have incorporated almost all the comments made by the referees.
	Here are the specific changes in response to the comments of the referees.
\\
\begin{itemize}
\item
	General comments: On the other hand, this proof has a weakness relative to the L-P-P proof: it does not show that $n P(Z_n > 0)$ converges, and uses that fact. 
	Maybe there is a way to deduce it from the analysis here, but I did not see it.
	
	{\it Yes, there is a limitation in our biasing approach. 
	In the revised version, in order to demonstrate this limitation, we added the following discussion at the beginning of Section 1.2:
	
	``Suppose that $X$ is a non-negative 
	random variable with $E[X] \in (0,\infty)$,
	then its distribution conditioned on $\{ X > 0\}$ can be characterized by its conditional expectation $E[X|X>0]$ and its size-biased transform $\dot X$.
	In fact, for each $\lambda \geq 0$,
	\begin{equation*}
	(1.4)
	\begin{split}
	&E[1-e^{-\lambda X}|X>0]
	= \frac{E[1-e^{-\lambda X}]}{P(X>0)}
	\\&\quad = \frac{1}{P(X>0)}\int_0^\lambda E[Xe^{-s X}]ds = E[X|X>0]\int_0^\lambda E[e^{-s \dot X}]ds.
	\end{split}
	\end{equation*}
	Conversely, the distribution of $\dot X$ can be characterized 
	%	by $E[X|X>0]$ and the distribution of $X$ conditioned on $\{ X > 0\}$: 
	by the distribution of $X$ conditioned on $\{ X > 0\}$: 
	\begin{equation*}
	(1.5)
\begin{split}
	E[e^{- \lambda \dot X}] 
	= \frac{E[Xe^{-\lambda X}| X>0]}{E[X| X>0]}.
	\end{split}
	\end{equation*}
	As a consequence,  
	Theorem 1.1	is equivalent to
	\begin{equation*}
	(1.6)
	\begin{split}
	E\big[\frac{Z_n}{n}| Z_n > 0\big]
	\xrightarrow[n\to \infty]{} \frac{\sigma^2}{2}
	\end{split}
	\end{equation*}
	and
	\[
	(1.7) 
	\begin{split}
	\label{eq: convergence after size-biased}
	E[e^{-s \frac{\dot Z_n}{n}}]
	\xrightarrow[n\to \infty]{} E[e^{-s \dot Y}],
	\end{split}
	\]
	where $\dot Y$ is a $Y$-transform 
	of the exponential random variable $Y$.
	Indeed, since $E[Z_n] = 1$, (1.6) is equivalent to Theorem 1.1(1); and under (1.6), according to (1.4) and (1.5), we can see (1.7) is equivalent to Theorem 1.1(2).
	The $k(k-1)$-type size-biased tree will be only used to derive result (1.7).
	In Section 3, for completeness, we will simplify 
	the argument of [2] and [9],
	and give a proof of Theorem 1.1(1)."
}
	
	{\it Also, in the revised version, pp11, at the beginning of Section 3, for completeness, we added an elementary proof to $n P(Z_n > 0) \xrightarrow[n\to \infty]{} 2/ \sigma^2$.}
\\	
\item[1.] 
	pp2, first paragraph: ``random element" should be ``random variable" (twice). 
	{\it Changed as suggested.}
\\	
\item[2.] 
	pp2, 3 lines from the end, ``followed-up paper" should be ``follow-up paper" (also at the bottom of pp5). 
	{\it Changed as suggested.}
\\
\item[3.] 
	pp3, line 3: delete ``on the exponential convergence". 
	{\it Changed as suggested.}
\\
\item[4.] 
	pp4, just after (1.5): To deduce the result from your heuristic, you seem to be assuming that looking at $\ddot Y$ is equivalent, or at least similar, to looking at the conditioned process.  
	Why is that true? 
	Is there any heuristic reason why this biasing should be similar to conditioning to survive? 
	
	{\it This is a good comment. 
	The mistake we made here is that the statement of our heuristic is not entirely true. 
	It is also not entirely true to think biasing is similar to conditioning. 
	More proper way is to only think they are highly related.
	In the revised version, pp4, at the beginning of Section 1.2, in order to demonstrate their relation, we added the following:

	``Suppose that $X$ is a non-negative 
	random variable with $E[X] \in (0,\infty)$,
	then its distribution conditioned on $\{ X > 0\}$ can be characterized by its conditional expectation $E[X|X>0]$ and its size-biased transform $\dot X$.
	In fact, for each $\lambda \geq 0$,
	\begin{equation*}
	(1.4)
	\begin{split}
	&E[1-e^{-\lambda X}|X>0]
	= \frac{E[1-e^{-\lambda X}]}{P(X>0)}
	\\&\quad = \frac{1}{P(X>0)}\int_0^\lambda E[Xe^{-s X}]ds = E[X|X>0]\int_0^\lambda E[e^{-s \dot X}]ds.
	\end{split}
	\end{equation*}
	Conversely, the distribution of $\dot X$ can be characterized 
	%	by $E[X|X>0]$ and the distribution of $X$ conditioned on $\{ X > 0\}$: 
	by the distribution of $X$ conditioned on $\{ X > 0\}$: 
	\begin{equation*}
	(1.5)
	\begin{split}
	E[e^{- \lambda \dot X}] 
	= \frac{E[Xe^{-\lambda X}| X>0]}{E[X| X>0]}."
	\end{split}
	\end{equation*}
	
	
	Also, in the revised version, pp5, just after (1.9), we changed our heuristic into the following:

	``Suppose that $\dot Z_n/n$ converges weakly to a random variable $\dot Y$, and $\ddot Z_n/n$ converges weakly to a random variable $\ddot Y$. 
	Then, according to [7, Lemma 4.3], $\ddot Y$ is a size-biased transform of $\dot Y$. 
	Therefore, letting $n\to\infty$ in (1.9), 
	$\dot Y$ should satisfy (1.8), which, by Lemma 1.3, suggests that (1.7) is true."
	
	We think this new statement is more accurate.}
\\
\item[5.] 
	pp4, ``It would be interesting to compare" should be ``It is interesting to compare". 
	{\it Changed as suggested.}
\\
\item[6.] 
	pp7, line 7, I think $C_u$ should be $\dot C_u$ here, twice. 
	{\it Changed as suggested.}
\\
\item[7.]
	pp8, proof of Thm 1.2, the first paragraph could be reduced to ``Note that $\{(X_m(t))_{0 \leq m \leq n}; G_n\} =^d (Z_m)_{0 \leq m \leq n}$ and $\{(X_m(t))_{0\leq m \leq n}; \dot G_n\} =^d (\ddot Z_n)_{0 \leq m \leq n}$." There is no need for so much detail.
	{\it Changed as suggested.}
\\	
\item[8.] 
	pp8, halfway down, "Using this analysis, one can verify that" should be "Using this analysis, we verify that". 
	{\it Changed as suggested.}
\\
\item[9.] 
	pp9, first display, there is no `` $d$ " required here, this is a discrete space, you can just write `` $\frac{\ddot G_n(t)}{G_n(t)}$ ". 
	{\it We agree. 
		There is no `` $d$ " required here. 
		Our change in the revised version is to delete this display, because if we dropped
		``$d$", it became a duplicate of (2.3).}
\\
\item[10.] 
	pp9, end of proof of Thm 1.2, the QED symbol should be after (2.5). 
	{\it Changed as suggested.}
\\
\item[11.] 
	pp9, start of section 2.2, ``notations" should be ``notation" (notation is a continuum, like water). Also change ``give the precise meaning that (1.5) admits a size-biased add-on structure for the size-biased $\mu$-Galton-Watson tree" to ``give a precise meaning to (1.5)". 
	{\it Changed as suggested.}
\\	
\item[12.]
	pp9, Proof of Prop 2.1, this is again a bit over-elaborate at the start. 
	Could write simply ``For any particle $u = u_1\dots u_n$", write $[\emptyset, u]:= \{u_1\dots u_j: j = 0,\dots, n\}$. 
	The particles in $\dot T$ can be separated according to their nearest spin ancestor; we write
\[
	\dot A_k = \{ u \in \dot T: |[\emptyset, u] \cap \dot V| = k\}.
\]
	Then
\[
	X_n(\dot T) = \sum_{k=0}^n X_n(\dot A_k). "
\]
	{\it Changed as suggested.}
\\	 
\item[13.]
	pp10, top: Change ``For convention" to ``and", also change ``transform" to ``transforms".
	{\it Changed as suggested.}
\\
\item[14.]
	pp10, paragraph after (2.7), $\ddot V' - \ddot V\cap \ddot V'$ should be $\ddot V'\setminus \ddot V$, and also ``is separated" should be ``can be separated".
	{\it Changed as suggested.}
\\
\item [15.]
	pp10, again the discussion before (2.8) can be shortened in the same way as at the start of the proof.
	{\it Changed as suggested.}	 
\\
\item [16.]
	pp10, (2.9): Not sure I see the second equality. 
	I think the last $E[e^{- \lambda \dot Z_m}]$ should be $E[e^{- \lambda \dot Z_{n-m}}]$. 
	What am I missing?
	Even if I'm right, this doesn't seem to cause major problems, but it does need to be chased through the rest of the document.
	{\it We think this equality is correct. 
	In the revised version, at the bottom of pp 10, we added an extra step to demonstrate this.}
\\
\item [17.]
	pp11, top, delete ``The proof is complete."
	{\it Changed as suggested.}
\\
\item [18.]
	pp11, halfway down, after ``Therefore" add ``since $x - \ln x$ is decreasing on $[0,1]$."
	{\it Changed as suggested.}
\\
\item [19.]
	pp11, bottom, change ``By dividing $\|F\|_\infty$ on the both side of $(3.1)$" to ``By dividing both sides of $(3.1)$ by $\|F\|_\infty$.
	{\it Changed as suggested.}
\\
\item [20.]
	pp11/12, Proof of Lemma 3.2, you can take $k = 0$ as the base case here, which is trivial. 
	Then your proof is shorter.
	{\it 
	In the revised version, at bottom of pp 12, we gave a new proof to Lemma 3.2, which is much better and much shorter. }
\\
\item [21.]
	pp12, halfway, change ``verify that, $e$ also satisfies" to ``verify that $e$ satisfies".
	{\it Changed as suggested.}
\\
\item[22.]
	pp12, end of proof of Lemma 1.3, you cite [1, Lemma 2.6], but this is a five-year-old preprint that does not appear to have been peer-reviewed. 
	Please provide a better reference that has been peer-reviewed.
	
	{\it In the revised version, pp 13, at the end of Proof of Lemma 1.3. In order to show $Y =^d e$ from $\dot Y=^d \dot e$ without [1, Lemma 2.6], we changed our statement into the following:
	
	``Since $Y$ and $\mathbf e$ are strictly positive, according to (1.4), we have
	\[
	E[1-e^{-\lambda Y}]/ E[Y] = E[1-e^{-\lambda \mathbf e}]/ E[\mathbf e], \quad \lambda \geq 0.
	\]
	Taking $\lambda \to \infty$, we get $E[Y] = E[\mathbf e]$. Therefore, $Y \overset{d} = \mathbf e$ as desired."
	
	Also, in the References, we deleted that old preprint.}
\\
\item [23.]
	pp13, first display, there should be a dot on $Z_{\lfloor un \rfloor}$.
	{\it Changed as suggested.}		
\end{itemize}
\end{document}
