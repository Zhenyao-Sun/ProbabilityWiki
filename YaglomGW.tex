%%%----Versions-----------------------------
%---2017/4/21 by Yanxia-----------------
%---2017/4/19 by Renming-----------------
%---2017/4/18 by Yanxia---------------
%---2017/4/16 by Zhenyao---------------
%---2017/4/11 by Renming-----------------
%---2017/4/11 by Yanxia-----------------
%---2017/4/6 by Zhenyao-----------------
%%%----Documentstyles-----------------------
\documentclass[12pt]{amsart}
\usepackage{geometry}
%\geometry{left=2.5cm,right=2.5cm,top=3cm,bottom=3cm}
\geometry{left=3cm,right=3cm,top=3cm,bottom=3cm}
%\usepackage[numbers]{natbib}
\usepackage{hyperref}
%%%----Environments------------------------
\newtheorem{thm}{Theorem}[section]
%\newtheorem{asp}[thm]{Assumption}
%\newtheorem{cor}[thm]{Corollary}
\newtheorem{lem}[thm]{Lemma}
\newtheorem{prop}[thm]{Proposition}
%\theoremstyle{remark}
%\newtheorem{rem}[thm]{Remark}
%\newtheorem{ex}[thm]{Example}
\numberwithin{equation}{section}
\newcommand{\defn}[1]{{\em #1}}
%%%----Notations---------------------------
\newcommand{\ind}[1]{\mathbf 1_{#1}}
%\newcommand{\filtration}{\mathcal}
\newcommand{\prob}{\mathbf P}
%\newcommand{\norm}[1]{\left\| #1 \right\|}
\newcommand{\abs}[1]{\left| #1 \right|}
\newcommand{\expr}[1]{\left( #1 \right)}
\newcommand{\brac}[1]{\left[ #1 \right]}
%\newcommand{\inner}[1]{\left\langle #1 \right\rangle}
\newcommand{\set}[1]{\left\{ #1 \right\}}
%\newcommand{\asure}{\text{-a.s. }}
%\newcommand{\aevery}{\text{-a.e. }}
%\newcommand{\suchthat}{\text{ s.t. }}
\newcommand{\law}{\text{\rm Law}}
\newcommand{\tolaw}{\stackrel{d}{\to}}
%\newcommand{\toas}{\stackrel{\text{a.s.}}{\to}}
\newcommand{\uniformly}{\text{uniformly in }}
\newcommand{\parent}[1]{\overleftarrow{#1}}
\newcommand{\tree}{\mathbf t}
\newcommand{\spine}{\mathbf v}
\newcommand{\ancestor}[1]{[\emptyset,#1]}
\newcommand{\expct}{\mathbf E}
\newcommand{\eqlaw}{\overset{d}{=}}
%\newcommand{\while}{\text{ while }}
%%%----Fonts--------------------------------
\newcommand{\bA}{\mathbf A}\newcommand{\bbA}{\mathbb A}\newcommand{\cA}{\mathcal A}
\newcommand{\bB}{\mathbf B}\newcommand{\bbB}{\mathbb B}\newcommand{\cB}{\mathcal B}
\newcommand{\bC}{\mathbf C}\newcommand{\bbC}{\mathbb C}\newcommand{\cC}{\mathcal C}
\newcommand{\bD}{\mathbf D}\newcommand{\bbD}{\mathbb D}\newcommand{\cD}{\mathcal D}
\newcommand{\bE}{\mathbf E}\newcommand{\bbE}{\mathbb E}\newcommand{\cE}{\mathcal E}
\newcommand{\bF}{\mathbf F}\newcommand{\bbF}{\mathbb F}\newcommand{\cF}{\mathcal F}
\newcommand{\bG}{\mathbf G}\newcommand{\bbG}{\mathbb G}\newcommand{\cG}{\mathcal G}
\newcommand{\bH}{\mathbf H}\newcommand{\bbH}{\mathbb H}\newcommand{\cH}{\mathcal H}
\newcommand{\bI}{\mathbf I}\newcommand{\bbI}{\mathbb I}\newcommand{\cI}{\mathcal I}
\newcommand{\bJ}{\mathbf J}\newcommand{\bbJ}{\mathbb J}\newcommand{\cJ}{\mathcal J}
\newcommand{\bK}{\mathbf K}\newcommand{\bbK}{\mathbb K}\newcommand{\cK}{\mathcal K}
\newcommand{\bL}{\mathbf L}\newcommand{\bbL}{\mathbb L}\newcommand{\cL}{\mathcal L}
\newcommand{\bM}{\mathbf M}\newcommand{\bbM}{\mathbb M}\newcommand{\cM}{\mathcal M}
\newcommand{\bN}{\mathbf N}\newcommand{\bbN}{\mathbb N}\newcommand{\cN}{\mathcal N}
\newcommand{\bO}{\mathbf O}\newcommand{\bbO}{\mathbb O}\newcommand{\cO}{\mathcal O}
\newcommand{\bP}{\mathbf P}\newcommand{\bbP}{\mathbb P}\newcommand{\cP}{\mathcal P}
\newcommand{\bQ}{\mathbf Q}\newcommand{\bbQ}{\mathbb Q}\newcommand{\cQ}{\mathcal Q}
\newcommand{\bR}{\mathbf R}\newcommand{\bbR}{\mathbb R}\newcommand{\cR}{\mathcal R}
\newcommand{\bS}{\mathbf S}\newcommand{\bbS}{\mathbb S}\newcommand{\cS}{\mathcal S}
\newcommand{\bT}{\mathbf T}\newcommand{\bbT}{\mathbb T}\newcommand{\cT}{\mathcal T}
\newcommand{\bU}{\mathbf U}\newcommand{\bbU}{\mathbb U}\newcommand{\cU}{\mathcal U}
\newcommand{\bV}{\mathbf V}\newcommand{\bbV}{\mathbb V}\newcommand{\cV}{\mathcal V}
\newcommand{\bW}{\mathbf W}\newcommand{\bbW}{\mathbb W}\newcommand{\cW}{\mathcal W}
\newcommand{\bX}{\mathbf X}\newcommand{\bbX}{\mathbb X}\newcommand{\cX}{\mathcal X}
\newcommand{\bY}{\mathbf Y}\newcommand{\bbY}{\mathbb Y}\newcommand{\cY}{\mathcal Y}
\newcommand{\bZ}{\mathbf Z}\newcommand{\bbZ}{\mathbb Z}\newcommand{\cZ}{\mathcal Z}
%%%---------Top matter----------------
\title[A 2-spine decomposition and Yaglom's theorem]
%{\large A 2-spine Decomposition on the Critical Galton-Watson Tree and a Probabilistic Proof of Yaglom's Theorem}
{\large A 2-spine Decomposition of the Critical Galton-Watson Tree and a Probabilistic Proof of Yaglom's Theorem}
%---------------------
\author{Yan-Xia Ren}
\address{
	Yan-Xia Ren\\
	School of Mathematical Sciences\\
	Peking University, Beijing\\
	P. R. China, 100871}
\email{yxren@math.pku.edu.cn}
\thanks{The research of the first author is supported in part by NSFC (Grant Nos. 11271030 and 11671017)}
%--------------------
\author{Renming Song}
\address{
	Renming Song\\
	Dept of Mathematics\\
	University of Illinois at Urbana-Champaign\\
	Urbana, IL 61801}
\email{rsong@illinois.edu}
\thanks{The research of the second author is supported in part by the Simons Foundation (\#429343, Renming Song)}
%--------------------
\author{Zhenyao Sun}
\address{
	Zhenyao Sun\\
	School of Mathematical Sciences\\
	Peking University\\
	Beijing, P. R. China, 100871}
\curraddr{
	Department of Mathematics\\
	University of Illinois at Urbana-Champaign\\
	Urbana, IL 61801}
\email{zhenyao@illinois.edu}
\thanks{The third author is supported by the China Scholarship Council.
%\new added
Corresponding author.
%end new
}
%----------------------------
\keywords{
	Galton-Watson process, Galton-Watson tree, spine decomposition, Yaglom's theorem, martingale change of measure}
\subjclass[2010]{60J80, 60F05}
\date{\today}
%%%----Main-------------------------------
\begin{document}
\begin{abstract}
	In this note  we propose a two-spine decomposition of the critical
		Galton-Watson
	tree and use this decomposition to give a probabilistic proof of Yaglom's theorem.
\end{abstract}
	\maketitle	
\section{Introduction}
\subsection{Model}
\label{sec:model}
	Consider a Galton-Watson process $((Z_n)_{n\ge0}; \bP )$ with offspring distribution $\mu=(\mu(n))_{n\ge 0}$.
	Let $L$ be a random variable with law $\mu$. Assume that $Z_0=1$ and the process is critical in the sense that
\begin{equation}
\label{eq:mean}
		\expct [L]
	=
		\sum_{k=0}^\infty k \mu(k)
	=
		1.
\end{equation}
	Suppose $L$ has finite variance
\begin{equation}
\label{eq:variance}
		0	
	<	
		\sigma^2
	:=
		\sum_{k=0}^\infty  (k-1)^2 \mu(k)
	=
		\sum_{k=0}^\infty k(k-1) \mu(k)
	<
		\infty.
\end{equation}
For simplicity, we will call a Galton-Watson process with offspring distribution $\mu$
a $\mu$-Galton-Watson process. 	It is well known that
\begin{thm}[\cite{kesten1966galton}]
\label{thm:kesten}
For a $\mu$-Galton-Watson process satisfying \eqref{eq:mean} and \eqref{eq:variance}, we have
\begin{enumerate}
\item
\label{thm:kolmogorov}
	$\displaystyle \lim_{n \to \infty} n \bP \set{Z_n>0} = \frac{2}{\sigma^2},$ 
\item
\label{thm:yaglom}
	$\displaystyle	\law\expr{\frac{Z_n}{n}\bigg| Z_n>0}\tolaw Y \ \text{as }n\to\infty,$
\end{enumerate}
	where $Y$ is an exponential random variable with mean $\frac{\sigma^2}{2}$.
\end{thm}
\par
	Under a third moment assumption, assertions \eqref{thm:kolmogorov} and \eqref{thm:yaglom} of Theorem \ref{thm:kesten} are due to \cite{kolmogorov1938losung} and \cite{yaglom1947certain} respectively.
Theorem \ref{thm:kesten}(2) is usually called Yaglom's theorem.
	For probabilistic proofs of the above results, we refer our readers to 
%\cite{lyons1995conceptual}, \cite{geiger1999elementary} and \cite{geiger2000new}.
\cite{geiger1999elementary}, \cite{geiger2000new} and \cite{lyons1995conceptual}.
\par
	In \cite{lyons1995conceptual}, Lyons, Pemantle and Peres gave a
probabilistic proof of Theorem \ref{thm:kesten} using the size-biased $\mu$-Galton-Watson tree.
	Denote by $\dot L$ a random variable with the \defn{size-biased distribution} of $L$, that is, for any bounded Borel function $f$,
\begin{equation*}
		\expct\brac{f\expr{\dot L}}
	=
		\frac{\expct\brac{Lf(L)}}{\expct\brac{L}}.
\end{equation*}
The celebrated \defn{size-biased $\mu$-Galton-Watson tree} is
then constructed as follows:
\begin{itemize}
\item
	There is an initial particle which is marked.
\item
	Any marked particle gives independent birth to a random number of children according to $\dot L$. Pick one of those children randomly as the new marked particle while leaving the other children as unmarked particles.
\item
	Any unmarked particle gives independent birth to a random number of unmarked children according to $L$.
\item
	The evolution goes on.
\end{itemize}
\par
	Notice that the marked particles form a descending family line which will be referred to as the \defn{spine}.
	Define $\dot Z_n$ as the population of the $n$th generation in the size-biased tree.
	It is proved in \cite{lyons1995conceptual} that the process $(\dot Z_n)_{n\ge 0}$ is a martingale transform of the process $(Z_n)_{n\ge 0}$ using the martingale $(Z_n)_{n\ge 0}.$
	That is, for any generation number $n$ and any bounded Borel function $g$ on
		$\bbN_0^{n} := \{0,1,\dots\}^n$,
\begin{equation}
\label{eq:htransformation}
				\expct\brac{g\expr{\dot Z_1,\dots,\dot Z_n}}
	=
                \frac{\expct\brac{Z_n g\expr{Z_1,\dots, Z_n}}}{\expct\brac{Z_n}}.
\end{equation}
\par
It is natural to consider probabilistic proofs of analogous results for more general
critical branching processes. Vatutin and  Dyakonova \cite{VD} gave a probabilistic proof of Theorem \ref{thm:kesten}(1) for multitype critical branching processes.
As far as we know, there is no probabilistic proof of Yaglom's theorem for
multitype critical branching processes. It seems that it is difficult to adapt
the probabilistic proofs in 
%\cite{lyons1995conceptual} and \cite{geiger2000new} 
\cite{geiger2000new} and \cite{lyons1995conceptual} 
for monotype branching processes to more general models, such as multitype
branching processes, branching Hunt processes and superprocesses.
In this note we will give a new probabilistic proof of Theorem \ref{thm:kesten}(2), and we hope this new proof can be generalized to more general measure-valued Markov processes in future work.

We first present a $k(k-1)$-type size-biased $\mu$-Galton-Watson tree equipped
with a two-spine skeleton.
	Denote by $\ddot L$ a random variable with the \defn{$k(k-1)$-type size-biased distribution} of $L$, that is, for any
		bounded function $f$ on $\bbN_0$,
\begin{equation*}
		\expct\brac{f\expr{\ddot L}}
	=
		\frac{\expct\brac{L(L-1)f(L)}}{\expct\brac{L(L-1)}}.
\end{equation*}
	Fix a generation number $n$ and pick a random generation number $K_n$ uniformly among
		$\{0,\dots,n-1\}$.
	The \defn{$k(k-1)$-type size-biased $\mu$-Galton-Watson tree with height $n$} is then defined as a particle system such that:
\begin{itemize}
\item
	There is an initial particle which is marked.
\item
	Before or after generation $K_n$, any marked particle gives independent birth to a random number of children according to $\dot L$.
	Pick one of those children randomly as the new marked particle while leaving the other children as unmarked particles.
\item
	The marked particle at generation $K_n$, however, gives independent birth to a random number of children according to $\ddot L$.
	Pick two different particles randomly among those children as the new marked particles while leaving the other children as unmarked particles.
\item
	Any unmarked particle gives independent birth to a random number of unmarked children according to $L$.
\item
	The system stops at generation $n$.
\end{itemize}
\par
	If we track all the marked particles, it is clear that they form a descending family line which splits into two separate lines after generation $K_n$.
	In other words, the family tree of all the marked particles can be considered as a \defn{two-spine skeleton} with $K_n$ as the last generation where the two spines are together.
\par	
	For any $0\le m \le n$, denote by $\ddot Z_m^{(n)}$ the population of the
	$m$th generation in the $k(k-1)$-type size-biased $\mu$-Galton-Watson tree with height $n$.
	The main reason for proposing such a model is that the process $(\ddot Z_m^{(n)})_{0\le m\le n}$ can be obtained from the process $(Z_m)_{0\le m\le n}$ by a change of measure via the random variable $Z_n(Z_n-1)$.
	This is made precise in the result below which will be proved in Section \ref{sec:spacesandmeasures}.
\begin{thm}
\label{thm:changeofmeasure}
	Let $(Z_m)_{m\ge 0}$ be a $\mu$-Galton-Watson process and
		$(\ddot Z_m^{(n)})_{0\le m\le n}$
	be the population of a $k(k-1)$-type size-biased $\mu$-Galton-Watson tree with height $n$.
Suppose that $\mu$ satisfies \eqref{eq:mean} and \eqref{eq:variance}.
Then, for any bounded Borel function $g$ on $\bbN^{n}_0$,
\begin{equation*}
				 \prob\brac{g\expr{\ddot Z_1^{(n)},\dots,\ddot Z_n^{(n)}}}
	=
				 \frac{\prob\brac{Z_n(Z_n-1) g\expr{Z_1,\dots, Z_n}}}{\prob\brac{Z_n(Z_n-1)}}.			
\end{equation*}
	\end{thm}
\par
	The idea of considering a branching particle system with more than one spine is not new.
	A particle system with $k$ spines  was constructed in \cite{harris2015many} and used in the  many-to-few formula for the $k$th moment of branching Markov processes and branching random walks. Inspired by \cite{harris2015many}, we use a
two-spine model to characterize the $k(k-1)$-type size-biased branching process.
We will give a new spine decomposition, see \eqref{eq:rawtwospinedecomposition}
below, and then use this decomposition to give a new probabilistic proof of
Theorem \ref{thm:kesten}(2).

\subsection{Application}
It was shown in \cite{lyons1995conceptual} that, for the size-biased
$\mu$-Galton-Watson tree, the population of the particles off the spine
(i.e., unmarked particles) could be considered as a branching process with independent immigrations.
	That is, besides the independent $\mu$-branching mechanism for each existing particles, an additional independent $(\dot L-1)$-distributed random number of particles will immigrate into the system at each generation $k\ge 1$.
	Therefore, the number of off-spine particles in the $n$th generation, $\dot Z_n-1$, has a natural decomposition,
\begin{equation}
\label{eq:rawdecomposition}
		\dot Z_n-1
	=
		\sum_{k=1}^{n}\dot Z_n^{(k)},
\end{equation}
	where $\dot Z^{(k)}_n$ is the number of particles in the $n$th generation whose oldest off-spine ancestors have generation number $k$.
\par	
	Notice that $\{\dot Z_n^{(k)}: 1\le k\le n\}$ are independent random variables, and $\dot Z_n^{(k)}\eqlaw Z'_{n-k}$ for each
		$1\le k\le n$, where $(Z'_n)_{n\ge 0}$ is a $\mu$-Galton-Watson process with initial population distributed according to $\dot L-1$.
	Denote by $\cL_X(\lambda):=\bE[e^{-\lambda X}]$, $\lambda \geq 0$, the Laplace  transform of
		any nonnegative random variable $X$.
	Then \eqref{eq:rawdecomposition} can be rewritten as
\begin{equation}
\label{eq:spinedecomposition}
		\cL_{\dot Z_n}(\lambda)
	=
		e^{-\lambda}\prod_{m=0}^{n-1} \cL_{Z'_m}(\lambda),
	\quad
		\lambda \geq 0.
\end{equation}
\par
	Analogously, we consider a natural decomposition of the population $\ddot Z_n^{(n)}$ according to the two-spine skeleton in the $k(k-1)$-type size-biased
$\mu$-Galton-Watson tree. This decomposition leads us to another
	characterization of the distribution of $\dot Z_n$. This is made precise in the following result which will be proved in Section \ref{sec:spinesdecomposition}.
\begin{prop}
\label{lem:twospinedecomposition}
	Let $(\dot Z_n)_{n\ge 0}$ be a size-biased $\mu$-Galton-Watson process.
	Let $(Z_n')_{n\ge 0}$ and $(Z_n'')_{n\ge 0}$ be $\mu$-Galton-Watson branching processes with initial population distributed according to $\dot L-1$ and $\ddot L-2$ respectively.
	Suppose that $\mu$ satisfies \eqref{eq:mean} and \eqref{eq:variance}.
	Then, for any $n\in \bbN_0$,
\begin{equation}
\label{eq:twospinedecomposition}
		-\ln \cL_{\dot Z_n}(\lambda)
	=
		\lambda+\sigma^2\sum_{m=0}^{n-1}\int_0^\lambda \frac{\cL_{Z''_m}(s)}{\cL_{Z'_m}(s)}\cL_{\dot Z_m}(s)ds,
	\quad
		\lambda \geq 0.
\end{equation}
\end{prop}
\par
	With this proposition, we can discuss the asymptotic behavior of the distribution of $\dot Z_n$.
	From the criticality of branching process $(Z'_n)_{n\ge 0}$ and $(Z''_n)_{n\ge 0}$, we have that $\prob\set{Z_n'=0}\to 1$ and $\prob\set{Z_n''=0}\to 1$ as $n\to\infty$.
	Therefore, using
\begin{equation*}
		\prob\set{Z_n''=0}
	\leq
		\cL_{Z_n''}(s)
	\leq
		\frac{\cL_{Z_n''}(s)}{\cL_{Z_n'}(s)}
	\leq
		\frac{1}{\cL_{Z_n'}(s)}
	\leq
		\frac{1}{\prob \set{Z_n'=0}},
	\quad
		s\geq 0,\, n\ge 0,
\end{equation*}
	we get that
\begin{equation}
\label{eq:uniformly}
	    \frac{\cL_{Z_n''}(s)}{\cL_{Z_n'}(s)}
	\to
	    1,
	\quad
				\uniformly s\in [0,\infty)\text{ as } n\to\infty.
\end{equation}
	Replacing $\lambda$ with $\frac{\lambda}{n}$ and letting $n\to\infty$ in \eqref{eq:twospinedecomposition},
    we see that, if $\frac{\dot Z_n}{n}$ has a weak limit as $n\to\infty$, say $\dot Y$, then $\dot Y$ satisfies
\begin{equation}
\label{equation:exponential}
		-\ln \cL_{\dot Y}(\lambda)
	=
		\sigma^2\int_0^1du\cdot\int_0^\lambda \cL_{\dot Y}(us)ds,
	\quad\lambda \geq 0.
\end{equation}
	Solving \eqref{equation:exponential}, we get that $\dot Y$ has the size-biased distribution of $Y$
%new added
with $Y$ being an exponential random variable with mean $\frac{\sigma^2}{2}$.
%end new
In Section \ref{sec:anewproofofyaglomslaw}, we will give a new probabilistic proof of Yaglom's theorem using Proposition \ref{lem:twospinedecomposition}.
\section{Trees and their decomposition}
\label{sec:preliminary}
\subsection{Spaces and measures}
\label{sec:spacesandmeasures}
	Consider \defn{particles} as elements in the space
\begin{equation*}
		\cU
	:=
		\set{\emptyset}\cup\bigcup_{k=1}^\infty \bbN^k.
\end{equation*}
	where
		$\bbN:=\{1,2,\dots\}$.
	Therefore elements in $\cU$ are of the form 213, which we read as the individual being the 3rd child of the 1st child of the 2nd child of the initial ancestor $\emptyset$.
	For two particles
		$u=u_1\dots u_n, v=v_1\dots v_m\in\cU$,
	$uv$ denotes the concatenated particle
		$uv:=u_1\dots u_nv_1\dots v_m$.
	We use the convention $u\emptyset = \emptyset u = u$
		and $u_1\dots u_n=\emptyset$ if $n=0$.
	For any particle
		$u:=u_1\dots u_{n-1}u_n$, we define its \defn{generation} as $\abs{u}:=n$ and its \defn{parent particle} as
		$\parent{u}:=u_1\dots u_{n-1}$.
	For any particle $u\in\cU$ and any subset $\mathbf a\subset\cU$, we define the
		\defn{number of children of $u$ in $\mathbf a$}
	as $l_u(\mathbf a) := \#\set{\alpha\in \mathbf a:\parent{\alpha}=u} $.
	We also define the \defn{height} of $\mathbf a$ as $|\mathbf a|:=\sup_{\alpha\in \mathbf a}|\alpha|$ and its \defn{population in the $n$th generation} as $X_n(\mathbf a):=\#\set{u\in \mathbf a:|u|=n}$.
	A \defn{tree} $\tree$ is defined as a subset of $\cU$ such that there exists a $\bbN_0$-valued sequence $(l_u)_{u\in \cU}$, indexed by $\cU$, satisfying
\begin{equation*}
		\tree
	=
                 \set{u_1\dots u_m\in \cU: m\ge 0, u_j\leq l_{u_1\dots u_{j-1}}, \forall  j=1,\dots,m}.
\end{equation*}
	A \defn{spine} $\spine$ on a  tree $\tree$ is defined as a sequence of particles
		 $\{v^{(k)}:k=0,1,\dots,|\tree|\}\subset\tree$
	such that $v^{(0)}=\emptyset$ and $\parent{v^{(k)}}=v^{(k-1)}$ for any
	$k=1,\dots, |\tree|$.
	In the case that $|\tree|=\infty$, we simply write $k=0,1,\dots$ as $k=0,1,\dots, |\tree|$.
\par
	Fix a generation number $n\in \bbN$. Define the following spaces:
\begin{itemize}
\item
	\defn{The space of trees with height no more than $n$},
\begin{equation*}
		\bbT_{\leq n}
	:=
				\set{\tree:\tree\text{ is a tree with }|\tree|\leq n}.
\end{equation*}
\item
	\defn {The space of $n$-height trees with one distinguishable spine},
\begin{equation*}
		\dot{\bbT}_n
	:=
				\set{(\tree,\spine):\tree \text{ is a tree with } |\tree|=n,
\spine \text{ is a spine on } \tree}.
\end{equation*}
	\item \defn{The space of $n$-height trees with two different distinguishable spines},
\begin{equation*}
		\ddot{\bbT}_n
	:=
		\set{(\tree,\spine,\spine'):(\tree,\spine)\in\dot\bbT_n,(\tree,\spine')\in\dot\bbT_n,\spine\neq\spine'}.
\end{equation*}
\end{itemize}
\par
	Let $(L_u)_{u\in\cU}$ be a collection of independent random variables with law $\mu$, indexed by $\cU$. Denote by $T$ the random tree defined by
\begin{equation*}
		T
	:=
		\set{u_1\dots u_m\in \cU: 0\le m\le n, u_j\leq L_{u_1\dots u_{j-1}},\forall j=1,\dots,m}.
\end{equation*}
We refer to $T$ as a \defn{$\mu$-Galton-Watson tree with height no more than n}
	since its population
		$(X_m(T))_{0\le m\le n}$
is a $\mu$-Galton-Watson process stopped at generation $n$.
	Define the \defn{$\mu$-Galton-Watson measure $\bG_n$} on $\bbT_{\leq n}$ as the law of random tree $T$. That is, for any $\tree\in\bbT_{\leq n}$,
\begin{equation*}
		\bG_n(\tree)
    :=
		\prob(T=\tree)
	=
        \prob(L_u=l_u(\tree)\text{ for any } u\in\tree \text{ with }|u|<n)
	=
		\prod_{u\in \tree:|u|<n}\mu(l_u(\tree)).
\end{equation*}
\par
	Recall that $\dot L$ has the size-biased distribution of $L$.
	Define $\dot C$ as a random number which, conditioned on $\dot L$, is uniformly distributed on
	$\{1,\dots,\dot L\}$.
	Independent of $(L_u)_{u\in\cU}$, let $(\dot L_u,\dot C_u)_{u\in \cU}$ be a collection of independent copies of $(\dot L,\dot C)$, indexed by $\cU$.
	We then use $(L_u)_{u\in\cU}$ and $(\dot L_u,\dot C_u)_{u\in\cU}$ as the building blocks to construct the size-biased
$\mu$-Galton-Watson tree $\dot T$ and its distinguishable spine $\dot V$
	following the steps described in Section \ref{sec:model}.
	We use $L_u$ as the number of children of particle $u$ if $u$ is unmarked and use $\dot L_u$ if $u$ is marked.
	In the latter case, we always set the $C_u$th child of $u$, i.e. particle $uC_u$, as the new marked particle.
	For convenience, we stop the system at generation $n$. To be precise, the random spine $\dot V$ is defined by
\begin{equation*}
		\dot V
	:=
		\set{v_1\dots v_m\in \cU:0\le m\le n, v_j=\dot C_{v_1\dots v_{j-1}},\forall j=1,\dots,m},
\end{equation*}
	and the random tree $\dot T$ is defined by
\begin{equation*}
		\dot T
	:=
		\set{u_1\dots u_m\in\cU: 0\le m\le n,u_j\leq \tilde L_{u_1\dots u_{j-1}},\forall j=1,\dots,m},
\end{equation*}
	where, for any $u\in\cU$, $\tilde L_u:=L_u\ind{u\not\in \dot V}+\dot L_u\ind{u\in \dot V}$.
\par
	We now consider the distribution of the $\dot\bbT_n$-valued random element $(\dot T,\dot V)$.
	For any $(\tree,\spine)\in\dot\bbT_n$, the event $\{(\dot T,\dot V)=(\tree,\spine)\}$ occurs if and only if:
\begin{itemize}
\item
    $L_u=l_u(\tree)$ for each $u\in \tree\setminus\spine$ with $\abs{u}<n$ and
\item
		$(\dot L_{v_1\dots v_m},\dot C_{v_1\dots v_m})=(l_{v_1\dots v_m}(\tree),v_{m+1})$ for each $v_1\dots v_{m+1}\in\spine$ with $0\le m\le n-1$.
\end{itemize}
    Therefore, the distribution of $(\dot T,\dot V)$ can be determined by
\begin{equation}
\label{eq:treespinemeasure}
		\prob((\dot T,\dot V)=(\tree,\spine))
	=
		\prod_{u\in \tree\setminus\spine:|u|<n}\mu(l_u(\tree))
	\cdot
		\prod_{u\in \spine:\abs{u}<n}l_u(\tree)\mu(l_u(\tree))\frac{1}{l_u(\tree)}
		=
		\bG_n(\tree).
\end{equation}
\par	
	The \defn{size-biased $\mu$-Galton-Watson measure $\dot \bG_n$} on
	$\bbT_{\leq n}$ is then defined as the law of the $\bbT_{\leq n}$-valued random element $\dot T$. That is, for any $\tree\in\bbT_{\leq n}$,
\begin{equation}
\label{eq:sizebiasedGWmeasure}
\begin{split}
		\dot \bG_n(\tree)
	&:=
		\bP(\dot T=\tree)
	=
		\sum_{\spine:(\tree,\spine)\in \dot\bbT_n} \bP((\dot T,\dot V)=(\tree,\spine))
	\\&=
	    \#\set{\spine:(\tree,\spine)\in \dot\bbT_n}
	\cdot
	    \bG_n(\tree)
	=
		X_n(\tree)
	\cdot
		\bG_n(\tree).
\end{split}
\end{equation}
\par
	Equations \eqref{eq:treespinemeasure}, \eqref{eq:sizebiasedGWmeasure} and their consequence \eqref{eq:htransformation} were first obtained in \cite{lyons1995conceptual}.
We use these equations to help us to understand how the $k(k-1)$-type size-biased $\mu$-Galton-Watson tree can be represented.
\par	
	Recall that $K_n$ is a random generation number uniformly distributed on
		$\{0,\dots,n-1\}$,
	and $\ddot L$ is a random variable with the $k(k-1)$-type size-biased distribution of $L$.
	Define $(\ddot C,\ddot C')$ as a random vector which, conditioned on $\ddot L$, is uniformly distributed on $\{(i,j)\in\bbN^2:1\leq i\neq j\leq \ddot L\}$.
	Suppose that $(L_u)_{u\in\cU}, (\dot L_u,\dot C_u)_{u\in \cU}$, $(\ddot L,\ddot C,\ddot C')$ and $K_n$ are independent of each other.
	We now use these elements to build the $k(k-1)$-type size-biased
$\mu$-Galton-Watson tree $\ddot T$ and
	its two different distinguishable spines $\ddot V$ and $\ddot V'$ following the steps described in Section \ref{sec:model}.
	Write $C_u:=\dot C_u\ind{|u|\neq K_n}+\ddot C\ind{|u|=K_n}$ and $C'_u:=\dot C_u\ind{|u|\neq K_n}+\ddot C'\ind{|u|=K_n}$.
	We define the random spines $\ddot V$ and $\ddot V'$ as
\begin{align*}
        \ddot V
	&:=
				\set{v_1\dots v_m\in \cU:0\le m\le n, v_j= C_{v_1\dots v_{j-1}},\forall j=1,\dots,m},
	\\
		\ddot V'
	&:=
				\set{v_1\dots v_m\in \cU:0\le m \le n, v_j= C'_{v_1\dots v_{j-1}},\forall j=1,\dots,m},
\end{align*}
	and the random tree $\ddot T$ as
\begin{equation*}
	    \ddot T
	:=
		\set{u_1\dots u_m\in\cU: 0\le m\le n,u_j\leq L''_{u_1\dots u_{j-1}},\forall j=1,\dots,m},
\end{equation*}
	where, for any $u\in\cU$, $L''_u:=L_u \ind{u\not\in \ddot V\cup\ddot V'}+\dot L_u \ind{u\in \ddot V\cup\ddot V',|u|\neq K_n}+\ddot L\ind{u\in \ddot V\cup\ddot V',|u|=K_n}$.
\par
We now consider the distribution of
	$(\ddot T,\ddot V,\ddot V')$. For any $(\tree,\spine,\spine')\in\ddot\bbT_n$, the event $\{(\ddot T,\ddot V,\ddot V')=(\tree,\spine,\spine')\}$ occurs if and only if:
\begin{itemize}
\item
    $K_n=k_n:=|\spine\cap\spine'|$,
\item
    $L_u=l_u(\tree)$ for each $u\in \tree\setminus(\spine\cup\spine')$ with $\abs{u}<n$,
\item
        $(\dot L_{v_1\dots v_m},\dot C_{v_1\dots v_m})=(l_{v_1\dots v_m}(\tree),v_{m+1})$ for each $v_1\dots v_mv_{m+1}\in\spine\cup\spine'$ with $k_n\neq m<n$ and
\item
        $(\ddot L,\ddot C,\ddot C')=(l_{v_1\dots v_{k_n}}(\tree),v_{k_n+1},v'_{k_n+1})$ for $v_1\dots v_{k_n}v_{k_n+1}\in\spine$ and $v_1\dots v_{k_n}v'_{k_n+1}\in\spine'$.
\end{itemize}
	Using this analysis, one can verify that
\begin{align*}
&\prob\left((\ddot T,\ddot V,\ddot V')=(\tree,\spine,\spine')\right)\\
&	= \frac{1}{n} \cdot
	    \prod_{u\in \tree\setminus(\spine\cup \spine'):|u|<n} \mu(l_u(\tree))
	\cdot
    	\prod_{u\in \spine\cup \spine':k_n\neq|u|<n}l_u(\tree)
    	\mu(l_u(\tree))\frac{1}{l_u(\tree)}
    \\
    &\quad\cdot
		\prod_{u\in \spine\cup \spine':|u|=k_n}\frac{l_u(\tree)(l_u(\tree)-1)
		\mu(l_u(\tree))}{\sigma^2}\frac{1}{l_u(\tree)(l_u(\tree)-1)}\\
	&=
		\frac{1}{n\sigma^2}
		\bG_n(\tree).
\end{align*}
\par	
The \defn{$k(k-1)$-type size-biased $\mu$-Galton-Watson measure
	$\ddot \bG_n$} on
	$\bbT_{\leq n}$ is then defined as the law of the random element $\ddot T$. That is, for any $\tree\in\bbT_{\leq n}$,
\begin{equation}
\label{eq:k(k-1)typesizebiasedGWmeasure}
\begin{split}
		\ddot \bG_n(\tree)
	&:=
		\bP(\ddot T=\tree)
	=
		\sum_{(\spine,\spine'):(\tree,\spine,\spine')\in \ddot\bbT_n}
		\bP\left((\ddot T,\ddot V,\ddot V')=(\tree,\spine,\spine')\right)
	\\&=
	    \#\set{(\spine,\spine'):(\tree,\spine,\spine')\in \ddot\bbT_n}
	\cdot
	    \frac{\bG_n(\tree)}{n\sigma^2}
	=
		\frac{X_n(\tree)(X_n(\tree)-1)}{n\sigma^2}
    \cdot
        \bG_n(\tree).
\end{split}
\end{equation}

We note in passing that, because of the way they are constructed, the measures
$(\ddot \bG_n)_{n\ge 1}$ are not consistent, that is, the measure $\ddot \bG_n$ is
not the restriction of $\ddot \bG_{n+1}$. This implies that the change of measure in
Theorem \ref{thm:changeofmeasure} is not a martingale change of measure.

\medskip
\begin{proof}[Proof of Theorem \ref{thm:changeofmeasure}]
Since
		$(X_m(\tree))_{0\le m\le n}$ under $\bG_n$ is exactly a
$\mu$-Galton-Watson process stopped at generation $n$, we have that
       $((X_m(\tree))_{0\le m\le n};\bG_n) \eqlaw (Z_m)_{0\le m\le n}.$
    Similarly, since the process
       $(X_m(\tree))_{0\le m\le n}$
    under measure $\ddot \bG_n$ is constructed as the population of a
$k(k-1)$-type size-biased $\mu$-Galton-Watson tree with height $n$, we have that
       $((X_m(\tree))_{0\le m\le n};\ddot \bG_n) \eqlaw (\ddot Z_m)_{0\le m\le n}.$
	Equation \eqref{eq:k(k-1)typesizebiasedGWmeasure} can be rewritten in the following form
\begin{equation*}
    	\frac{\ddot \bG_n(d\tree)}{\bG_n(d\tree)}
    =
    	\frac{X_n(\tree)(X_n(\tree)-1)}{n\sigma^2},
    \quad
    	\tree\in\bbT_{\leq n}.
\end{equation*}
    For any bounded Borel function $g$ on $\bbN_0^n$, we can then verify that
\begin{multline}
\label{eq:proofofchangeofmeasure}
              \expct\brac{g\expr{\ddot Z_1^{(n)},\dots,\ddot Z_n^{(n)}}}
	=
                \ddot \bG_n\brac{g\expr{X_1(\tree),\dots, X_n(\tree)}}
    \\=
                \bG_n\brac{\frac{X_n(\tree)(X_n(\tree)-1)}{n\sigma^2}g\expr{X_1(\tree),\dots, X_n(\tree)}}
	=
				\frac{\expct\brac{Z_n(Z_n-1) g\expr{Z_1,\dots,Z_n}}}{n\sigma^2}.
\end{multline}
\end{proof}
	Taking $g\equiv 1$ in equation \eqref{eq:proofofchangeofmeasure}, we get that $\expct[Z_n(Z_n-1)]=n\sigma^2$.
	
	
\subsection{Spine decomposition}
\label{sec:spinesdecomposition}
	For any particle $u=u_1\dots u_n$, we define
\begin{equation*}
	    \ancestor{u}
	:=
	    	    \set{u_1\dots u_j:j=0,\dots, n}
\end{equation*}
	as the \defn{descending family line from $\emptyset$ to $u$}.
Consider the size-biased $\mu$-Galton-Watson tree $(\dot T,\dot V)$.
	Since $|\dot V|=n$, we must have, for any particle $u\in\cU$,
	$|\ancestor{u}\cap\dot V|\in\set{0,1,\dots,n}$.
	Therefore, the particles in $\dot T$ can be separated into $n+1$ parts,
\begin{equation*}
		\dot T
	=
		\bigcup_{k=0}^n\dot A_k
	:=
	    	    \bigcup_{k=0}^n\set{u\in\dot T:\abs{\ancestor{u}\cap \dot V}=k}.
\end{equation*}
	This gives a natural decomposition of the $n$th generation population of $\dot T$,
\begin{equation}
\label{eq:generationseperation}
		X_n(\dot T)
	=
		\sum_{k=0}^nX_n(\dot A_k).
\end{equation}
	Equation \eqref{eq:generationseperation} is exactly the spine decomposition \eqref{eq:rawdecomposition} in the sense that
\begin{equation*}
                \expr{X_n(\dot T),(X_n(\dot A_k))_{0\le k\le n-1},X_n(\dot A_n)}
    \eqlaw
               \expr{\dot Z_n,(\dot Z_n^{(k+1)})_{0\le k\le n-1},1}.
\end{equation*}
\par
	Similarly, consider the $k(k-1)$-type size-biased
$\mu$-Galton-Watson tree $(\ddot T,\ddot V,\ddot V')$.
	Since $|\ddot V\cup\ddot V'|=n$, we must have, for any particle $u\in\cU$, $|\ancestor{u}\cap(\ddot V\cup \ddot V')|\in\{0,1,\dots,n\}$.
	Therefore, the particles in $\ddot T$ can also be separated into $n+1$ parts,
\begin{equation*}
		\ddot T
	=
		\bigcup_{k=0}^n \ddot A_k
	:=
		\bigcup_{k=0}^n\set{u\in\ddot T: \abs{\ancestor{u}\cap(\ddot V\cup\ddot V')}=k}.
\end{equation*}
	This gives a natural decomposition of the $n$th generation population of $\ddot T$,
\begin{equation}
\label{eq:rawtwospinedecomposition}
		X_n(\ddot T)
	=
		\sum_{k=0}^nX_n(\ddot A_k)
	=
	    \sum_{k=0}^{K_n-1}X_n(\ddot A_k)
	+
		X_n(\ddot A_{K_n})
	+
		\sum_{k=K_n+1}^nX_n(\ddot A_k).
\end{equation}
	Conditioning on $K_n=m$ with
	$m\in\{0,\dots,n-1\}$,
	we get from the construction of $(\ddot T,\ddot V,\ddot V')$ that
\begin{itemize}
\item
        $X_n(\ddot A_k)\eqlaw Z'_{n-k-1}$ for any $0\le k\le m-1$,
\item
    $X_n(\ddot A_m)\eqlaw Z''_{n-m-1}$,
\item
    $\sum_{k=m+1}^nX_n(\ddot A_k)\eqlaw \dot Z_{n-m-1}^{(1)}+\dot Z_{n-m-1}^{(2)}$ and
\item
      $\{X_n(\ddot A_k):0\le k\le n\}$
    are independent of each other.
\end{itemize}
	Here, $(Z_k')_{k\ge 0}$ and $(Z_k'')_{k\ge 0}$ are $\mu$-Galton-Watson processes with initial population distributed according to $\dot L-1$ and $\ddot L-2$ respectively. $(\dot Z_k^{(1)})_{k\ge 0}$ and $(\dot Z_k^{(2)})_{k\ge 0}$ are two independent copies of the size-biased $\mu$-Galton-Watson process $(\dot Z_k)_{k\ge 0}$.
	Following those arguments, taking Laplace transform, we can rewrite \eqref{eq:rawtwospinedecomposition} as,
\begin{equation}
\label{eq:laplacetransformationoftwospinedecomposition}
\begin{split}
		\cL_{\ddot Z_n^{(n)}}(\lambda)
	&=
		\prob\brac{e^{-\lambda X_n(\ddot T)}}
	=
	     \sum_{m=0}^{n-1}\prob(K_n=m)\prob\brac{e^{-\lambda X_n(\ddot T)}\big| K_n=m}
	\\&=
        \frac{1}{n}\sum_{m=0}^{n-1}\prod_{k=0}^{m-1}\cL_{Z'_{n-k-1}}(\lambda)
    \cdot
        \cL_{Z''_{n-m-1}}(\lambda)\cL_{\dot Z_{n-m-1}}^2(\lambda)
    ,\quad\lambda\geq 0.
\end{split}
\end{equation}
\medskip
\begin{proof}[Proof of Proposition \ref{lem:twospinedecomposition}]
    It follows from \eqref{eq:spinedecomposition} that, for any $\lambda\geq 0$,
\begin{multline}
\label{eq:keytakingback}
	    \frac{\cL_{\dot Z_n}(\lambda)}{\cL_{Z_{n-m-1}'}(\lambda)\cL_{\dot Z_{n-m-1}}(\lambda)}
	=
	    \cL_{Z_{n-m-1}'}^{-1}(\lambda)\frac{e^{-\lambda}\prod_{m=0}^{n-1}\cL_{Z'_m}(\lambda)}{e^{-\lambda}\prod_{m=0}^{n-m-2}\cL_{Z'_m}(\lambda)}
	=
	    \prod_{k=0}^{m-1}\cL_{Z'_{n-k-1}}(\lambda).
\end{multline}
	Plugging \eqref{eq:keytakingback} into \eqref{eq:laplacetransformationoftwospinedecomposition}, we get that, for any $\lambda\geq 0$,
\begin{equation}
\label{eq:keyrelation}
        \cL_{\ddot Z_n^{(n)}}(\lambda)
	=
        \frac{1}{n}\sum_{m=0}^{n-1}\cL_{\dot Z_n}(\lambda)\frac{\cL_{Z''_{n-m-1}}(\lambda)}{\cL_{Z_{n-m-1}'}(\lambda)}\cL_{\dot Z_{n-m-1}}(\lambda)
    =
        \frac{1}{n}\cL_{\dot Z_n}(\lambda)\sum_{m=0}^{n-1}\frac{\cL_{Z''_m}(\lambda)}{\cL_{Z'_m}(\lambda)}\cL_{\dot Z_m}(\lambda).
\end{equation}
    According to \eqref{eq:htransformation} and Theorem \ref{thm:changeofmeasure}, $\dot Z_n$ and $\ddot Z_n^{(n)}$ have the size-biased and $k(k-1)$-type size-biased distribution of $Z_n$ respectively. Therefore
\begin{equation}
\label{eq:firstderivative}
        \cL_{\dot Z_n}(\lambda)
	=
	    \expct\brac{Z_n e^{-\lambda Z_n}}
	=
	    -\cL_{Z_n}'(\lambda)
  ,\quad \lambda > 0,
\end{equation}
    and
\begin{equation}
\label{eq:secondderivative}
        \cL_{\ddot Z_n}(\lambda)
	=
		\frac{1}{n\sigma^2}\expct\brac{Z_n(Z_n-1)e^{-\lambda Z_n}}
	=
		\frac{1}{n\sigma^2}\expr{\cL_{Z_n}''(\lambda)+\cL_{Z_n}'(\lambda)},
		\quad \lambda > 0.
\end{equation}
	It then follows from \eqref{eq:keyrelation}, \eqref{eq:firstderivative} and \eqref{eq:secondderivative} that
\begin{equation*}
		-\frac{d}{d\lambda}\ln \cL_{\dot Z_n}(\lambda)
	=
		-\frac{\cL_{Z_n}''(\lambda)}{\cL_{Z_n}'(\lambda)}
	=
		1
	+
		\sigma^2\sum_{m=0}^{n-1}\frac{\cL_{Z''_{m}}(\lambda)}{\cL_{Z'_{m}}(\lambda)}\cL_{\dot Z_{m}}(\lambda),
	\quad
		\lambda > 0.
\end{equation*}
	Integrating from $0$ to $\lambda$ gives that, for any $\lambda\geq 0$,
\begin{equation*}
		-\ln \cL_{\dot Z_n}(\lambda)
	=
		\lambda
	+
		\sigma^2\sum_{m=0}^{n-1}\int_0^\lambda \frac{\cL_{Z''_{m}}(s)}{\cL_{Z'_{m}}(s)}\cL_{Z_m}(s)ds.
\end{equation*}
\end{proof}
\section{A new proof of Yaglom's theorem}
\label{sec:anewproofofyaglomslaw}
\begin{lem}
\label{lem:zeroinequality}
    Suppose that $c>0$ is a constant, and $F:[0,\infty)\to [0,1]$  is
    a function satisfying that, for any $\lambda\geq 0$,
\begin{equation}
\label{eq:zeroinequality}
	    F(\lambda)
	\leq
	    \frac{1}{c}\int_0^1du
	\cdot
	    \int_0^\lambda F(us)ds.
\end{equation}
    Then $F\equiv 0$.
\end{lem}
\begin{proof}
	We first show that $F(\lambda)=0$ for $\lambda \in [0,c)$.
    Using  $F(us)\leq 1$ on the right hand of \eqref{eq:zeroinequality} gives us
\begin{equation*}
        F(\lambda)
    \leq
        \frac{1}{c}\int_0^1du
    \cdot
	    \int_0^\lambda ds
	=
	    \frac{\lambda}{c},
	\quad
		\lambda\geq 0.
\end{equation*}
	Plugging this new upper bound into the right hand of \eqref{eq:zeroinequality} gives an updated upper bound of $F$:
\begin{equation*}
        F(\lambda)
    \leq
        \frac{1}{c}\int_0^1du
    \cdot
	    \int_0^\lambda \frac{us}{c}ds
	\leq
        \frac{1}{c}\int_0^1du
    \cdot
	    \int_0^\lambda \frac{\lambda}{c}ds
	=
	    \expr{\frac{\lambda}{c}}^2,
	\quad
		\lambda\geq 0.
\end{equation*}
    Repeating this process, we will have $F(\lambda)\leq (\frac{\lambda}{c})^m$ for any $m>0$. Therefore $F=0$ on $[0,c)$.
\par
    To complete the proof, we then show that, for any $k\in\bbN$, $F=0$ on $[0,kc)$ implies that $F=0$ on $[0,(k+1)c)$.
	Actually, since $F=0$ on $[0,kc)$, we have
\begin{equation*}
	    F\expr{kc+\lambda}
	\leq
		\frac{1}{c}\int_0^1 du\cdot\int_0^{kc+\lambda}F(us)ds
	=
		\frac{1}{c}\int_0^1du\cdot\int_{kc}^{kc+\lambda}
	    F(us)ds\leq\frac{\lambda}{c}, \quad \lambda\geq 0.
\end{equation*}
	Iterating, we get that
\begin{equation*}
	    F(kc+\lambda)
	\leq
	   	    \frac{1}{c}\int_0^1du\cdot\int_{kc}^{kc+\lambda} F(us)ds
	\leq
	    	    \frac{1}{c}\int_0^1du\cdot\int_{kc}^{kc+\lambda} \frac{\lambda}{c}ds
	\leq
		\expr{\frac{\lambda}{c}}^2, \quad \lambda\geq 0.
\end{equation*}
	Repeating this process gives that $F(kc+\lambda)\leq (\frac{\lambda}{c})^m$ for any $m>0$. Therefore $F=0$ on $[0,(k+1)c)$. The rest of the proof follows by induction.
\end{proof}
\medskip
\begin{proof}[Proof of Theorem \ref{thm:kesten}\eqref{thm:yaglom}.]
	Write $\nu:=\frac{2}{\sigma^2}$. Define
\begin{equation*}
		M(a,\lambda)
	:=
		\abs{\cL_{\frac{\dot Z_{[a]}}{a}}(\lambda)-
		\frac{\nu^2}{(\nu+\lambda)^2}},
	\quad
		\lambda \geq 0, a\geq 0.
		\end{equation*}
	We first show that, for any $\lambda\geq 0$,
\begin{equation}
\label{eq:Miszerofunction}
		M(\lambda)
	:=
		\limsup_{a\to\infty}M(a,\lambda)
	=
		0.
\end{equation}
	Actually, replacing $\lambda$ with $\frac{\lambda}{n}$ in \eqref{eq:twospinedecomposition}, we have that, for any $\lambda\geq 0$,
\begin{equation}
\label{eq:passingtolimitequation}
\begin{split}
		-\ln \cL_{\frac{\dot Z_n}{n}}(\lambda)
	&=
		\frac{\lambda}{n}+\sigma^2\sum_{m=0}^{n-1}\int_0^{\frac{\lambda}{n}} \frac{\cL_{Z''_m}(s)}{\cL_{Z'_m}(s)}\cL_{\dot Z_{m}}(s)ds
	\\&=
		\frac{\lambda}{n}+\frac{2}{\nu}\int_0^1 du \cdot \int_0^{\lambda} \frac{\cL_{Z''_{[un]}}\expr{\frac{s}{n}}}{\cL_{Z'_{[un]}}\expr{\frac{s}{n}}}\cL_{\frac{\dot Z_{[un]}}{un}}(us)ds.
\end{split}
\end{equation}
	On the other hand, we have by elementary calculus that
\begin{equation}
\label{eq:limitequation}
        -\ln\frac{\nu^2}{(\nu+\lambda)^2}
    =
		\frac{2}{\nu}\int_0^1du\cdot\int_0^\lambda \frac{\nu^2}{(\nu+us)^2}ds,
\quad \lambda\geq 0.
\end{equation}
	Using \eqref{eq:passingtolimitequation}, \eqref{eq:limitequation} and the obvious fact that $\abs{a-b}\leq\abs{\ln a-\ln b}$ for any $0<a,b\leq 1$, we conclude that, for any $\lambda\geq 0$,
\begin{equation}
\label{eq:inequalitybeforepassingtolimit}
\begin{split}
        M(n,\lambda)
    &\leq
		\abs{\ln \cL_{\frac{\dot Z_n}{n}}(\lambda)-\ln\frac{\nu^2}{(\nu+\lambda)^2}}
	\\&\leq
		\frac{\lambda}{n}
	+
	    \frac{2}{\nu}\int_0^1 du
	\cdot
	    \int_0^{\lambda} \abs{\frac{\cL_{Z''_{[un]}}\expr{\frac{s}{n}}}{\cL_{Z'_{[un]}}\expr{\frac{s}{n}}}\cL_{\frac{\dot Z_{[un]}}{un}}(us)- \frac{\nu^2}{(\nu+us)^2}}ds.
\end{split}
\end{equation}
    Taking $\limsup_{n\to\infty}$ in
    \eqref{eq:inequalitybeforepassingtolimit}, using \eqref{eq:uniformly} and the reverse Fatou's lemma, we arrive at
\begin{equation*}
	    M(\lambda)
    \leq
        \frac{2}{\nu}\int_0^1du
    \cdot
    	\int_0^\lambda M(us)ds,
    \quad
    	\lambda\geq 0.
\end{equation*}
	Thus by Lemma \ref{lem:zeroinequality} we get that $M\equiv 0$.
\par
	The rest of the proof follows a standard tightness argument.
For the $\mu$-Galton-Watson process $((Z_n)_{n\ge 0};\prob)$,
	write $\prob_n^*[\cdot]:=\prob[\cdot|Z_n>0]$.
By Theorem \ref{thm:kesten}(1), we have
\begin{equation*}
	    \expct_n^*\brac{\frac{Z_n}{n}}
	=
		 \expct\brac{\frac{1_{Z_n>0}}{\prob(Z_n>0)}\frac{Z_n}{n}}
	=
	   	     \frac{\expct\brac{Z_n}}{n\prob(Z_n>0)}
    \to
        \frac{1}{\nu}
   \quad
				\text{as }n\to\infty.
\end{equation*}
	As a consequence, $(\frac{Z_n}{n};\prob_n^*)_{n\ge 1}$ is tight and there is some sub-sequence, say $(\frac{Z_{n_k}}{n_k};\prob_{n_k}^*)_{k\ge 1}$, converging in law to some non-negative random variable, say $Y$.
		For any $\lambda > 0$,
	from the fact that the function $x\mapsto 1_{x\geq 0}xe^{-\lambda x}$ is bounded and continuous, we have
\begin{equation}
\label{eq:subsequence}
	    \expct_{n_k}^*\brac{\frac{Z_{n_k}}{n_k}\exp\set{-\lambda\frac{Z_{n_k}}{n_k}}}
	\to
	    \expct\brac{Ye^{-\lambda Y}},
	\quad
				\text{as }k\to\infty.
\end{equation}
	On the other hand, using \eqref{eq:htransformation}, we see that, for each $n\in\bbN$,
\begin{equation}
\label{eq:sizebiasedandcondition}
	    \expct\brac{e^{-\lambda\frac{\dot Z_n}{n}}}
	=
	    \expct\brac{Z_n e^{-\lambda\frac{Z_n}{n}}}
	=
		n\prob\set{Z_n>0}\expct_n^*\brac{\frac{Z_n}{n}e^{-\lambda\frac{Z_n}{n}}},
		\quad \lambda > 0.
	\end{equation}
	Taking $n=n_k$ and letting $k\to\infty$, it follows from
Theorem \ref{thm:kesten}(1),
	\eqref{eq:subsequence} and \eqref{eq:sizebiasedandcondition} that,
		for any $\lambda >0$,
	\begin{equation*}
	    \expct\brac{e^{-\lambda\frac{\dot Z_{n_k}}{n_k}}}
	\to
		\nu\expct\brac{Ye^{-\lambda Y}}
		\quad
		\text{as } k\to \infty.
	\end{equation*}
	Then by \eqref{eq:firstderivative} and \eqref{eq:Miszerofunction}, we have, for any $\lambda >0$,
\begin{equation*}
	    -\nu\cL_Y'(\lambda)
	=
	    \nu\expct\brac{Ye^{-\lambda Y}}
    =	
		\frac{\nu^2}{(\nu+\lambda)^2},
\end{equation*}
	which says $\cL_Y'(\lambda)=-\frac{\nu}{(\nu+\lambda)^2}$.
	Integration then gives that
\begin{equation*}
	    \cL_Y(\lambda)
	=
	    1
	+
	    \int_0^\lambda\frac{-\nu}{(\nu+s)^2}ds
	=
		\frac{\nu}{\nu+\lambda},
		\quad
		\lambda \ge 0.
	\end{equation*}
	So, $Y$ must be an 	exponential random variable
	with mean $\nu^{-1}=\frac{\sigma^2}{2}$.
	In particular, the distribution of $Y$
	is independent of the choice of $n_k$, and hence we have convergence in law for the whole sequence, as desired.
\end{proof}
%%%------Bibliography-------------------
\vspace{.1in}
\begin{thebibliography}{99}

\bibitem{AN} Athreya, K.  and  Ney, P.: Functionals of critical multitype branching processes, {\it Ann.
Probab.} {\bf 2} (1974), 339--343.

\bibitem{geiger1999elementary}
	Geiger, J.:
	Elementary new proofs of classical limit theorems for Galton--Watson processes.
	{\it J. Appl. Probab.} \textbf{36} (1999), 310--309.
\bibitem{geiger2000new}
	Geiger, J.:
	A new proof of Yaglom's exponential limit law.
	In {\it Mathematics and Computer Science}, pp. 245--249.
	Springer, 2000.
\bibitem{harris2015many}
	Harris, S. and Roberts, M.:
	The many-to-few lemma and multiple spines.
	{\it Ann.  Inst. Henri Poincar{\'e}, Probab. Stat.}
	\textbf{53} (2017), 226--242.
\bibitem{kesten1966galton}
	Kesten, H.,  Ney, P and Spitzer, F.:
	The Galton-Watson process with mean one and finite variance.
	{\it Theory Probab. Appl.}
	\textbf{11} (1966), 513--540.
\bibitem{kolmogorov1938losung}
	Kolmogorov, A. N.:
	Zur l{\"o}sung einer biologischen aufgabe.
	{\it Comm. Math. Mech. Chebyshev Univ. Tomsk}
	\textbf{2} (1938), 1--12.
\bibitem{lyons1995conceptual}
	Lyons, R.,  Pemantle, R. and Peres, Y.:
	Conceptual Proofs of $ L \log L $ criteria for mean behavior of branching processes.
	{\it Ann. Probab.} \textbf{23} (1995), 1125--1138.

\bibitem{VD} Vatutin, V. A. and Dyakonova,  E. E.: The survival probability of a critical mutitype Galton-Watson branching process.
%Proceedings of the Seminar on Stability Problems for Stochastic Models, Part II (Nal\c{e}czow, 1999).
{\it J.  Math. Sci.} \textbf{106} (2001), 2752--2759.

\bibitem{yaglom1947certain}
	Yaglom, A. M:
	Certain limit theorems of the theory of branching random processes.
	{\it Doklady Akad. Nauk SSSR (NS)} \textbf{56} (1947), 795--798.
\end{thebibliography}
\end{document}
