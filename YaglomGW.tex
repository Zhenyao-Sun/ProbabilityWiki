%%%----Versions-----------------------------
%---2017/4/11 by Yanxia-----------------
%---2017/4/6 by Zhenyao Sun-----------------
%%%----Documentstyles-----------------------
\documentclass[12pt]{amsart}
\usepackage{geometry}
\geometry{left=2.5cm,right=2.5cm,top=3cm,bottom=3cm}
\usepackage[numbers]{natbib}
\usepackage{hyperref}
%%%----Environments------------------------
\newtheorem{thm}{Theorem}[section]
\newtheorem{asp}[thm]{Assumption}
\newtheorem{cor}[thm]{Corollary}
\newtheorem{lem}[thm]{Lemma}
\newtheorem{prop}[thm]{Proposition}
\theoremstyle{remark}
\newtheorem{rem}[thm]{Remark}
\newtheorem{ex}[thm]{Example}
\numberwithin{equation}{section}
\newcommand{\defn}[1]{{\em #1}}
%%%----Notations---------------------------
\newcommand{\ind}[1]{\mathbf 1_{#1}}
\newcommand{\filtration}{\mathcal}
\newcommand{\prob}{\mathbf P}
\newcommand{\norm}[1]{\left\| #1 \right\|}
\newcommand{\abs}[1]{\left| #1 \right|}
\newcommand{\expr}[1]{\left( #1 \right)}
\newcommand{\brac}[1]{\left[ #1 \right]}
\newcommand{\inner}[1]{\left\langle #1 \right\rangle}
\newcommand{\set}[1]{\left\{ #1 \right\}}
\newcommand{\asure}{\text{-a.s. }}
\newcommand{\aevery}{\text{-a.e. }}
\newcommand{\suchthat}{\text{ s.t. }}
\newcommand{\law}{\text{\rm Law}}
\newcommand{\tolaw}{\stackrel{d}{\to}}
\newcommand{\toas}{\stackrel{\text{a.s.}}{\to}}
\newcommand{\uniformly}{\text{uniformly in }}
\newcommand{\parent}[1]{\overleftarrow{#1}}
\newcommand{\tree}{\mathbf t}
\newcommand{\spine}{\mathbf v}
\newcommand{\ancestor}[1]{[\emptyset,#1]}
\newcommand{\expct}{\mathbf E}
\newcommand{\eqlaw}{\overset{d}{=}}
\newcommand{\while}{\text{ while }}
%%%----Fonts--------------------------------
\newcommand{\bA}{\mathbf A}\newcommand{\bbA}{\mathbb A}\newcommand{\cA}{\mathcal A}
\newcommand{\bB}{\mathbf B}\newcommand{\bbB}{\mathbb B}\newcommand{\cB}{\mathcal B}
\newcommand{\bC}{\mathbf C}\newcommand{\bbC}{\mathbb C}\newcommand{\cC}{\mathcal C}
\newcommand{\bD}{\mathbf D}\newcommand{\bbD}{\mathbb D}\newcommand{\cD}{\mathcal D}
\newcommand{\bE}{\mathbf E}\newcommand{\bbE}{\mathbb E}\newcommand{\cE}{\mathcal E}
\newcommand{\bF}{\mathbf F}\newcommand{\bbF}{\mathbb F}\newcommand{\cF}{\mathcal F}
\newcommand{\bG}{\mathbf G}\newcommand{\bbG}{\mathbb G}\newcommand{\cG}{\mathcal G}
\newcommand{\bH}{\mathbf H}\newcommand{\bbH}{\mathbb H}\newcommand{\cH}{\mathcal H}
\newcommand{\bI}{\mathbf I}\newcommand{\bbI}{\mathbb I}\newcommand{\cI}{\mathcal I}
\newcommand{\bJ}{\mathbf J}\newcommand{\bbJ}{\mathbb J}\newcommand{\cJ}{\mathcal J}
\newcommand{\bK}{\mathbf K}\newcommand{\bbK}{\mathbb K}\newcommand{\cK}{\mathcal K}
\newcommand{\bL}{\mathbf L}\newcommand{\bbL}{\mathbb L}\newcommand{\cL}{\mathcal L}
\newcommand{\bM}{\mathbf M}\newcommand{\bbM}{\mathbb M}\newcommand{\cM}{\mathcal M}
\newcommand{\bN}{\mathbf N}\newcommand{\bbN}{\mathbb N}\newcommand{\cN}{\mathcal N}
\newcommand{\bO}{\mathbf O}\newcommand{\bbO}{\mathbb O}\newcommand{\cO}{\mathcal O}
\newcommand{\bP}{\mathbf P}\newcommand{\bbP}{\mathbb P}\newcommand{\cP}{\mathcal P}
\newcommand{\bQ}{\mathbf Q}\newcommand{\bbQ}{\mathbb Q}\newcommand{\cQ}{\mathcal Q}
\newcommand{\bR}{\mathbf R}\newcommand{\bbR}{\mathbb R}\newcommand{\cR}{\mathcal R}
\newcommand{\bS}{\mathbf S}\newcommand{\bbS}{\mathbb S}\newcommand{\cS}{\mathcal S}
\newcommand{\bT}{\mathbf T}\newcommand{\bbT}{\mathbb T}\newcommand{\cT}{\mathcal T}
\newcommand{\bU}{\mathbf U}\newcommand{\bbU}{\mathbb U}\newcommand{\cU}{\mathcal U}
\newcommand{\bV}{\mathbf V}\newcommand{\bbV}{\mathbb V}\newcommand{\cV}{\mathcal V}
\newcommand{\bW}{\mathbf W}\newcommand{\bbW}{\mathbb W}\newcommand{\cW}{\mathcal W}
\newcommand{\bX}{\mathbf X}\newcommand{\bbX}{\mathbb X}\newcommand{\cX}{\mathcal X}
\newcommand{\bY}{\mathbf Y}\newcommand{\bbY}{\mathbb Y}\newcommand{\cY}{\mathcal Y}
\newcommand{\bZ}{\mathbf Z}\newcommand{\bbZ}{\mathbb Z}\newcommand{\cZ}{\mathcal Z}
%%%---------Top matter----------------
\title{A Two-spines Decomposition on the Critical Galton-Watson Trees}
%---------------------
\author{Yanxia Ren}
\address{
	Yanxia Ren\\
	School of Mathematical Sciences\\
	Peking University, Beijing\\
	P. R. China, 100871}
\email{yxren@math.pku.edu.cn}
\thanks{[Supporter for Y. Ren]}
%--------------------
\author{Renming Song}
\address{
	Renming Song\\
	Department of Mathematics\\
	University of Illinois at Urbana-Champaign\\
	Urbana, IL 61801}
\email{rsong@math.uiuc.edu}
\thanks{[Supporter for R. Song]}
%--------------------
\author{Zhenyao Sun}
\address{
	Zhenyao Sun\\
	School of Mathematical Sciences\\
	Peking University\\
	Beijing, P. R. China, 100871}
\curraddr{
	Department of Mathematics\\
	University of Illinois at Urbana-Champaign\\
	Urbana, IL 61801}
\email{zhenyao.sun@gmail.com}
\thanks{[Supporter for Z. Sun]}
%----------------------------
\keywords{[keywords]}
\subjclass[2010]{[...]}
\date{\today}
%%%----Main-------------------------------
\begin{document}
\begin{abstract}
	[abstract]
\end{abstract}
	\maketitle	
\section{Introduction}
\subsection{Model}
\label{sec:model}
%	Consider the Galton-Watson branching process $(Z_n)_{n=0..}$
Consider a Galton-Watson branching process $((Z_n)_{n\ge0}; \bP )$
with offspring distribution $\mu$. Let $L$ be a random variable with law $\mu$. Assume that $Z_0=1$, and the process is critical in the sense that
\begin{equation}
\label{eq:mean}
		m:
	=
		\expct [L]
	=
		\sum_{k=0}^\infty k \mu\set{k}
	= 1.
\end{equation}
	Suppose $L$ has a non-degenerate finite variance
\begin{equation}
\label{eq:variance}
		0	
	<	
		\sigma^2
	:=
		\sum_{k=0}^\infty  (k-1)^2 \mu\set{k}
	=
		\sum_{k=0}^\infty k(k-1) \mu\set{k}
	<
		\infty.
\end{equation}
	It's well known that
\begin{thm}[\citet*{kesten1966galton}]
\label{thm:kesten}
	For the Galton-Watson branching process
%$(Z_n)_{n=0..}$
$(Z_n)_{n\ge 0}$
%similarly changes will be done without using % below!
with offspring distribution $\mu$ satisfying \eqref{eq:mean} and \eqref{eq:variance}, we have:
\begin{enumerate}
\item
\label{thm:kolmogorov}
	Kolmogorov's estimate:
\begin{equation*}
		\lim_{n \to \infty} n \bP \set{Z_n>0}
	=
		\frac{2}{\sigma^2}.
\end{equation*}
\item
\label{thm:yaglom}
	Yaglom's law:
\begin{equation*}
		\law\expr{\frac{Z_n}{n}\bigg| Z_n>0}
	\tolaw
		Y
	%, \expr{n\to\infty},
\quad\mbox{as }n\to\infty,
\end{equation*}
	where $Y$ is an exponentially distributed random variable with mean $\frac{\sigma^2}{2}$.
\end{enumerate}
\end{thm}
\par
	Under a third moment assumption, \eqref{thm:kolmogorov} and \eqref{thm:yaglom} of Theorem \ref{thm:kesten} are due to \citet*{kolmogorov1938losung} and \citet*{yaglom1947certain} respectively. For probabilistic proofs using different methods, we refer readers to \citet*{lyons1995conceptual} and \citet*{geiger1999elementary}, \citet*{geiger2000new}.
\par
	\citet*{lyons1995conceptual}
%gives
gave
a conceptual proof of Theorem \ref{thm:kesten} using the size-biased Galton-Watson tree. Denote by $\dot L$ a random variable with \defn{size-biased distribution} of $L$, that is, for any bounded Borel function $f$,
\begin{equation*}
		\expct\brac{f\expr{\dot L}}
	=
		\frac{\expct\brac{Lf(L)}}{\expct\brac{L}}.
\end{equation*}
	The celebrated \defn{size-biased Galton-Watson tree} is then constructed as follows:
\begin{itemize}
\item
	There is an initial particle which is marked.
\item
	Any marked particle gives independent birth to a random number of children according to $\dot L$. Pick one of those children randomly as new marked particle while leaving other children as unmarked particles.
\item
	Any unmarked particle gives independent birth to a random number of unmarked children according to $L$.
\item
	The system goes on and on. %and on.
\end{itemize}
\par
	Notice all the marked particles form a descending family line which will be referred as the \defn{spine}. Define $\dot Z_n$ as the population of the $n$th generation in the size-biased tree. \citet*{lyons1995conceptual} proved that process $(\dot Z_n)_{n\ge 0}$ is the Doob's $h$-transformation of process $(Z_n)_{n=0..}$ with respect to the martingale $(Z_n)_{n\ge 0}.$ That is, for any generation number $n$ and a bounded Borel function $g$ on $\bbN_0^{n},$
\begin{equation}
\label{eq:htransformation}
		%\expct\brac{g\expr{\dot Z_1..\dot Z_n}}
          \expct\brac{g\expr{\dot Z_1,\cdots,\dot Z_n}}
	=
	%	\frac{\expct\brac{Z_n g\expr{Z_1..Z_n}}}{\expct\brac{Z_n}}.
        \frac{\expct\brac{Z_n g\expr{Z_1,\cdots, Z_n}}}{\expct\brac{Z_n}}.
\end{equation}
\par
	In this article, we present a $k(k-1)$-type size-biased Galton-Watson tree whose structure relies on a two-spines skeleton. Denote by $\ddot L$ a random variable with the \defn{$k(k-1)$-type size-biased distribution} of $L$, that is, for any bounded Borel function $f$,
\begin{equation*}
		\expct\brac{f\expr{\ddot L}}
	=
		\frac{\expct\brac{L(L-1)f(L)}}{\expct\brac{L(L-1)}}.
\end{equation*}
	Fix a generation number $n$ and pick a random generation number $K_n$ uniformly among
%$\{0,..,n-1\}$.
$\{0,\cdots,n-1\}$.
The \defn{$k(k-1)$-type size-biased $\mu$-Galton-Watson tree with height $n$} is then defined as a particle system such that:
\begin{itemize}
\item
	There is an initial particle which is marked.
\item
	Before or after generation $K_n$, any marked particle gives independent birth to a random number of children according to $\dot L$. Pick one of those children randomly as new marked particle while leaving other children as unmarked particles.
\item
	Any marked particle at generation $K_n$, however, gives an independent birth to a random number of children according to $\ddot L$. Pick two different particles randomly among those children as new marked particles while leaving other children as unmarked particles.
\item
	Any unmarked particle gives independent birth to a random number of unmarked children according to $L$.
\item
	The system stops at generation $n$.
\end{itemize}
\par
	If we track all the marked particles, it's clear that they form a descending family line which splits into two separated lines after generation $K_n$. In other words, the family tree of all the marked particles can be considered as a \defn{two-spines skeleton} with $K_n$ as the last generation at which those two spines {\bf coincide ?} with each other.
\par	
	The main reason of promoting such a model is that, if one write $\ddot Z_m^{(n)}$ for the population of the
%$m$th
$m$th ($1\le m\le n$)
generation in the $k(k-1)$-type size-biased tree with height $n$, there is a change of measure relationship between process
%$(\ddot Z_m^{(n)})_{m=0..n}$ and $(Z_m)_{m=0..n}$
$(\ddot Z_m^{(n)})_{m=0,\cdots,n}$ and $(Z_m)_{m=0,\dots,n}$
 according to random variable $Z_n(Z_n-1)$.
\begin{thm}[See proof in section \ref{sec:spacesandmeasures}]
\label{thm:changeofmeasure}
	Let $(Z_m)_{m\ge 0}$ be the $\mu$-Galton-Watson branching process and $(\ddot Z_m^{(n)})_{m=0,\cdots,n}$ be the population of $k(k-1)$-type size-biased $\mu$-Galton-Watson tree with height $n$. Suppose that the offspring distribution $\mu$ satisfies \eqref{eq:mean} and \eqref{eq:variance}. Then, for any bounded Borel function $g$ on $\bbN^{n}_0$,
\begin{equation*}
		\prob\brac{g\expr{\ddot Z_1^{(n)},\cdots,\ddot Z_n^{(n)}}}
	=
		\frac{\prob\brac{Z_n(Z_n-1) g\expr{Z_1,\cdots, Z_n}}}{\prob\brac{Z_n(Z_n-1)}},
\end{equation*}
%new added
where we take $\bbN_0=\{0, 1,2,\cdots\}$.
%end new
\end{thm}
\par
%	The idea of branching particle system with more than one spine is not new.
The idea of considering branching particle system with more than one spine is not new.
A related construction in  other situation occurs in \citet*{harris2015many}. They developed a branching particle system with $k$-spines as the key element to state a $k$th moment formula for branching Markov processes and branching random walks known as the many-to-few formula. Though highly inspired by their work, we employ a different two-spines model for the purpose to characterize the $k(k-1)$-type size-biased branching process.
\subsection{Application}	
	\citet*{lyons1995conceptual} argued that, in the size-biased Galton-Watson tree, the population of the particles off the spine (i.e. unmarked particles) could be considered as a branching process with independent immigrations. That is, besides the independent $\mu$-branching mechanism for each existing particles, an additional independent $(\dot L-1)$-distributed random number of particles will immigrate into the system at each generation $k\in\bbN$. Therefore, the number of off-spine particles in the $n$th generation, $\dot Z_n-1$, has a natural decomposition,
\begin{equation}
\label{eq:rawdecomposition}
		\dot Z_n-1
	=
		\sum_{k=1}^{n}\dot Z_n^{(k)},
\end{equation}
	where $\dot Z^{(k)}_n$ is the number of particles in the $n$th generation whose oldest off-spine ancestors have generation number $k$.
\par	
	Notice that $\{\dot Z_n^{(k)}: k=1,\cdots,n\}$ are independent random variables and $\dot Z_n^{(k)}\eqlaw Z'_{n-k}$ for each $k=1..n$ where $(Z'_n)_{n=0..}$ represent a $\mu$-Galton-Watson branching process with initial population distributed according to $\dot L-1$. Denote by
%$\cL_X(\lambda):=\bE[e^{-\lambda X}],(\lambda \geq 0)$
$\cL_X(\lambda):=\bE[e^{-\lambda X}](\lambda \geq 0),$
the Laplace  transformation of some nonnegative random variable $X$. We prefer to rewrite equation \eqref{eq:rawdecomposition} into
\begin{equation}
\label{eq:spinedecomposition}
		\cL_{\dot Z_n}(\lambda)
	=
		e^{-\lambda}\prod_{m=0}^{n-1} \cL_{Z'_m}(\lambda),
	%,(\lambda \geq 0).
\quad \lambda \geq 0.
\end{equation}
\par
	 As an analogy, we consider a natural decomposition of the population $\ddot Z_n^{(n)}$ according to the two-spines skeleton in the $k(k-1)$-type size-biased Galton-Watson tree. This decomposition leads us to an additional characterization of $\dot Z_n$'s distribution.
\begin{prop}[See proof in section \ref{sec:spinesdecomposition}]
\label{lem:twospinedecomposition}
	Let $(\dot Z_n)_{n\ge 0}$ be the size-biased $\mu$-Galton-Watson branching process. Let $(Z_n')_{n\ge 0}$ and $(Z_n'')_{n\ge 0}$ be $\mu$-Galton-Watson branching processes with initial population distributed according to $\dot L-1$ and $\ddot L-2$ respectively. Suppose that $\mu$ satisfies \eqref{eq:mean} and \eqref{eq:variance}. Then, for any $n\in \bbN_0$,
\begin{equation}
\label{eq:twospinedecomposition}
		-\ln \cL_{\dot Z_n}(\lambda)
	=
		\lambda+\sigma^2\sum_{m=0}^{n-1}\int_0^\lambda \frac{\cL_{Z''_m}(s)}{\cL_{Z'_m}(s)}\cL_{\dot Z_m}(s)ds,
%\expr{\lambda \geq 0}.
\quad \lambda \geq 0.
\end{equation}
\end{prop}
\par
	With this proposition, we then discuss the asymptotic behaviour of $\dot Z_n$'s distribution. From the criticality of branching process $(Z'_n)_{n\ge 0}$ and $(Z''_n)_{n\ge 0}$, we have that $\prob\set{Z_n'=0}\to 1$ and $\prob\set{Z_n''=0}\to 1$ while $n\to\infty$. Therefore, by
\begin{equation*}
		\prob\set{Z_n''=0}
	\leq
		\cL_{Z_n''}(s)
	\leq
		\frac{\cL_{Z_n''}(s)}{\cL_{Z_n'}(s)}
	\leq
		\frac{1}{\cL_{Z_n'}(s)}
	\leq
		\frac{1}{\prob \set{Z_n'=0}}
	%,(s\geq 0,n=0..),
,\quad s\geq 0,\, n\ge 0,
\end{equation*}
    we get that
\begin{equation}
\label{eq:uniformly}
	    \frac{\cL_{Z_n''}(s)}{\cL_{Z_n'}(s)}
	\to
	    1
	%,(\uniformly s\in [0,\infty)\while n\to\infty).
,\quad \uniformly s\in [0,\infty)\while n\to\infty.
\end{equation}
    Replacing $\lambda$ with $\frac{\lambda}{n}$ and passing $n\to\infty$ in equation \eqref{eq:twospinedecomposition}, we see that, if $\frac{\dot Z_n}{n}$ has a weak limit as $n\to\infty$, say $\dot Y$, then $\dot Y$ satisfies
\begin{equation}
\label{equation:exponential}
		-\ln \cL_{\dot Y}(\lambda)
	=
		\sigma^2\int_0^1du\cdot\int_0^\lambda \cL_{\dot Y}(us)ds
	%,\expr{\lambda \geq 0}.
.\quad\forall\lambda \geq 0.
\end{equation}
	Solving equation \eqref{equation:exponential}, we get that $\dot Y$ has the size-biased distribution of $Y$.

 This argument gives us a conceptual explanation of Yaglom's law, \eqref{thm:yaglom} of Theorem \ref{thm:kesten}, in a size-biased view.
%In section \ref{sec:anewproofofyaglomslaw}, we'll discuss this weak convergence and
In Section \ref{sec:anewproofofyaglomslaw}, we will
give a new probabilistic proof of Yaglom's law using Proposition \ref{lem:twospinedecomposition}.
\section{Trees and their decomposition}
\label{sec:preliminary}
\subsection{Spaces and measures}
\label{sec:spacesandmeasures}
	Consider \defn{particles} as elements in the space
\begin{equation*}
		\cU
	:=
		\set{\emptyset}\cup\bigcup_{k=1}^\infty \bbN^k.
\end{equation*}
%new added
where we take $\bbN=\{1,2,\cdots\}$. For two particles $u, v\in\cU$, $uv$ denotes the concatenated particle $(u\emptyset = \emptyset u = u)$, and therefore
 contains elements like ��213�� (or ��$\emptyset 213$��), which we read as ��the
individual being the 3rd child of the 1st child of the 2nd child of the initial ancestor $\emptyset$��.
%end new
	For any particle
%$u:=\expr{u_1..u_{n-1}u_n}$,
$u:=\expr{u_1\cdots u_{n-1}u_n}$,
define its \defn{generation} as $\abs{u}:=n$ and its \defn{parent particle} as
%$\parent{u}:=(u_1..u_{n-1})$.
$\parent{u}:=(u_1\cdots u_{n-1})$.
For any particle $u\in\cU$ and any subset $\mathbf a\subset\cU$, define the \defn{decedent number of $u$ in $\mathbf a$} as $l_u(\mathbf a) := \#\set{\alpha\in \mathbf a:\parent{\alpha}=u} $. We also define $\mathbf a$'s \defn{height} as $\norm{\mathbf a}:=\sup_{\alpha\in \mathbf a}|\alpha|$ and its \defn{population in the $n$th generation} as $X_n(\mathbf a):=\#\set{u\in \mathbf a:|u|=n}$. A \defn{planer tree} $\tree$ is defined as a subset of $\cU$ such that there exists a $\bbN_0$-valued sequence $(l_u)_{u\in \cU}$, indexed by $\cU$, satisfying
\begin{equation*}
		\tree
	=
		%\set{(u_1..u_m)\in \cU: m\in \bbN_0, u_j\leq l_{(u_1..u_{j-1})}, \forall j=1..m}.
          \set{(u_0\cdots u_m)\in \cU: m\in \bbN_0, u_j\leq l_{(u_0u_1\cdots u_{j-1})}, \forall j=1,\cdots, m}.
\end{equation*}
	A \defn{spine} $\spine$ on a planer tree $\tree$ is defined as a sequence of particles
%$\{v^{(k)}:k=1..\|\tree\|\}\subset\tree$
$\{v^{(k)}:k=0,1,\cdots,\|\tree\|\}\subset\tree$
such that $v^{(0)}=\emptyset$ and $\parent{v^{(k)}}=v^{(k-1)}$ for any $k=1,\cdots,\|\tree\|$. In the case that $\|\tree\|=\infty$, we simply write $k=1,\cdots,\|\tree\|$ for $k\in\bbN$.
\par
	Fix a generation number $n\in \bbN$. Define the following spaces:
\begin{itemize}
\item
	\defn{The space of planer trees with height no more than $n$},
\begin{equation*}
		\bbT_{\leq n}
	:=
		\set{\tree:\tree\text{ is a planer tree with }\norm{\tree}\leq n}.
\end{equation*}
\item
	\defn {The space of $n$-height planer trees with one distinguishable spine,}
\begin{equation*}
		\dot{\bbT}_n
	:=
		\set{(\tree,\spine):\tree \text{ is a planer tree with }\norm{\tree}=n, \spine \text{ is a spine on } \tree}.
\end{equation*}
	\item \defn{The space of $n$-height planer trees with two different distinguishable spines,}
\begin{equation*}
		\ddot{\bbT}_n
	:=
		\set{(\tree,\spine,\spine'):(\tree,\spine)\in\dot\bbT_n,(\tree,\spine')\in\dot\bbT_n,\spine\neq\spine'}.
\end{equation*}
\end{itemize}
\par
	Let $(L_u)_{u\in\cU}$ be a collection of independent random variables with law $\mu$, indexed by $\cU$. Denote by $T$ the random planer tree defined by
\begin{equation*}
		T
	:=
		%\set{(u_1..u_m)\in \cU: m=0..n, u_j\leq L_{(u_1..u_{j-1})},\forall j=1..m}.
        \set{(u_0u_1\cdots u_m)\in \cU: m=0,\cdots,n, u_j\leq L_{(u_0\cdots u_{j-1})},\forall j=1,\cdots,m}.
\end{equation*}
	We refer to $T$ as a \defn{Galton-Watson tree with height no more than n} since its population $(X_m(T))_{m=0,\cdots,n}$ is a Galton-Watson branching process stopped at generation $n$. Define the \defn{$\mu$-Galton-Watson measure $\bG_n$} on $\bbT_{\leq n}$ as the law of random planer tree $T$. That is, for any $\tree\in\bbT_{\leq n}$,
\begin{equation*}
		\bG_n\set{\tree}
	%=
    :=
		\prob\set{T=\tree}
	=
		%\prob\set{L_u=l_u(\tree),(\forall u\in\tree,|u|<n)}
        \prob\set{L_u=l_u(\tree),\,\forall u\in\tree,|u|<n}
	=
		\prod_{u\in \tree:|u|<n}\mu\set{l_u(\tree)}.
\end{equation*}
\par
	Recall that $\dot L$ has the size-biased distribution of $L$. Define $\dot C$ as a random number which, while conditioning on $\sigma(\dot L)$, is uniformly distributed on $\{1,\cdots,\dot L\}$. Independent of $(\dot L_u)_{u\in\cU}$, let $(\dot L_u,\dot C_u)_{u\in \cU}$ be a collection of independent copies of $(\dot L,\dot C)$, indexed by $\cU$.  We then use $(L_u)_{u\in\cU}$ and $(\dot L_u,\dot C_u)_{u\in\cU}$ as the building elements to construct the size-biased Galton-Watson tree $\dot T$ and its distinguishable spine $\dot V$ following the steps described in section \ref{sec:model}. We use $L_u$ as the number of children for particle $u$ if $u$ is unmarked and use $\dot L_u$ if $u$ is marked. In the latter case, we always set the $C_u$th children of $u$, i.e. particle $(u,C_u)$, as the new marked particle. For our own interests, we stop the system at generation $n$. To be precise, the random spine $\dot V$ is defined by
\begin{equation*}
		\dot V
	:=
		%\set{(v_1..v_m)\in \cU:m=0..n, v_j=\dot C_{(v_1..v_{j-1})},\forall j=1..m},
         \set{(v_0\cdots v_m)\in \cU:m=0,\cdots,n, v_j=\dot C_{(v_0\cdots v_{j-1})},\forall j=1,\cdots,m},
\end{equation*}
	and the random planer tree $\dot T$ is defined by
\begin{equation*}
	%\dot T:=\set{(u_1..u_m)\in\cU: m=0..n,u_j\leq \tilde L_{(u_1..u_{j-1})},\forall j=1..m},
      \dot T:=\set{(u_0\cdots u_m)\in\cU: m=0,\cdots,n,u_j\leq \tilde L_{(u_0\cdots u_{j-1})},\forall j=1,\cdots,m},
\end{equation*}
	where, for any $u\in\cU$, $\tilde L_u:=L_u\ind{u\not\in \dot V}+\dot L_u\ind{u\in \dot V}$.
\par
	We then consider the distribution of the $\dot\bbT_n$-valued random element $(\dot T,\dot V)$. For any $(\tree,\spine)\in\dot\bbT_n$, the event $\{(\dot T,\dot V)=(\tree,\spine)\}$ occurs if and only if:
\begin{itemize}
\item
    $L_u=l_u(\tree)$ for each $u\in \tree\setminus\spine$ with $\abs{u}<n$ and
\item
    %$(\dot L_{(v_1..v_m)},\dot C_{(v_1..v_m)})=(l_{(v_1..v_m)}(\tree),v_{m+1})$ for each $(v_1..v_{m+1})\in\spine$ with $m=0..n-1$.
    $(\dot L_{(v_0\cdots v_m)},\dot C_{(v_0\cdots v_m)})=(l_{(v_0\cdots v_m)}(\tree),v_{m+1})$ for each $(v_0\cdots v_{m+1})\in\spine$ with $m=0,\cdots,n-1$.
\end{itemize}
    Therefore, the distribution of $(\dot T,\dot V)$ can be determined by
\begin{equation}
\label{eq:treespinemeasure}
		\prob\set{(\dot T,\dot V)=(\tree,\spine)}
	=
	    \prod_{u\in \tree\setminus\spine:|u|<n}\mu\set{l_u(\tree)}
	\cdot
		\prod_{u\in \spine:\abs{u}<n} \frac{l_u(\tree)\mu\set{l_u(\tree)}}{m}\frac{1}{l_u(\tree)}
   % =
   =:
		\bG\set{\tree}.
\end{equation}
\par	
	The \defn{size-biased Galton-Watson measure $\dot \bG_n$} on space $\bbT_{\leq n}$ is then defined as the law of the $\bbT_{\leq n}$-valued random element $\dot T$. That is, for any $\tree\in\bbT_{\leq n}$,
\begin{equation}
\label{eq:sizebiasedGWmeasure}
\begin{split}
		\dot \bG_n\set{\tree}
	&:=
		\bP\set{\dot T=\tree}
	=
		\sum_{\spine:(\tree,\spine)\in \dot\bbT_n} \bP\set{(\dot T,\dot V)=(\tree,\spine)}
	\\&=
	    \#\set{\spine:(\tree,\spine)\in \dot\bbT_n}
	\cdot
	    \bG_n\set{\tree}
	=
		X_n(\tree)
    \cdot
        \bG_n\set{\tree}.
\end{split}
\end{equation}
\par	
	Equation \eqref{eq:treespinemeasure}, \eqref{eq:sizebiasedGWmeasure} and their natural corollary \eqref{eq:htransformation} were first obtained by \citet*{lyons1995conceptual}. We recover them as a helpful analogy for understanding how the $k(k-1)$-type size-biased Galton-Watson tree can be represented.
\par	
	Recall that $K_n$ is a random generation number uniformly distributed on
%$\{0..n-1\}$,
$\{0,\cdots,n-1\}$,
and $\ddot L$ is a random variable with the $k(k-1)$-type size-biased distribution of $L$. Define $(\ddot C,\ddot C')$ as a random vector which, while conditioning on $\sigma(\ddot L)$, is uniformly distributed on $\{(i,j)\in\bbN^2:1\leq i\neq j\leq \ddot L\}$. Suppose that $(\dot L_u)_{u\in\cU}, (\dot L_u,\dot C_u)_{u\in \cU}$, $(\ddot L,\ddot C,\ddot C')$ and $K_n$ are independent of each other. We then use these elements to build the $k(k-1)$-type size-biased Galton-Watson tree $\ddot T$ and its two different distinguishable spines $\ddot V$ and $\ddot V'$ following the steps described in section \ref{sec:model}. Write $C_u:=\dot C_u\ind{|u|\neq K_n}+\ddot C\ind{|u|=K_n}$ and $C'_u:=\dot C_u\ind{|u|\neq K_n}+\ddot C'\ind{|u|=K_n}$. We define the random spines $\ddot V$ and $\ddot V'$ as
\begin{align*}
        \ddot V
	&:=
		%\set{(v_1..v_m)\in \cU:m=0..n, v_j= C_{(v_1..v_{j-1})},\forall j=1..m},
        \set{(v_0\cdots v_m)\in \cU:m=0,\cdots,n, v_j= C_{(v_0\cdots v_{j-1})},\forall j=1,\cdots,m},
	\\
	    \ddot V'
	&:=
		\set{(v_0\cdots v_m)\in \cU:m=0,\cdots,n, v_j= C'_{(v_0\cdots v_{j-1})},\forall j=1,\cdots,m},
\end{align*}
    and the random planer tree $\ddot T$ as
\begin{equation*}
	    \ddot T
	:=
	    %\set{(u_1..u_m)\in\cU: m=0..n,u_j\leq L''_{(u_1..u_{j-1})},\forall j=1..m},
          \set{(u_0\cdots u_m)\in\cU: m=0,\cdots,n,u_j\leq L''_{(u_0\cdots u_{j-1})},\forall j=1,\cdots,m},
\end{equation*}
	where, for any $u\in\cU$, $L''_u:=L_u \ind{u\not\in \ddot V\cup\ddot V'}+\dot L_u \ind{u\in \ddot V\cup\ddot V',|u|\neq K_n}+\ddot L\ind{u\in \ddot V\cup\ddot V',|u|=K_n}$.
\par
    We then consider the distribution of the random element $(\ddot T,\ddot V,\ddot V')$. For any $(\tree,\spine,\spine')\in\ddot\bbT_n$, the event $\{(\ddot T,\ddot V,\ddot V')=(\tree,\spine,\spine')\}$ occurs if and only if:
\begin{itemize}
\item
    $K_n=k_n:=\norm{\spine\cap\spine'}$,
\item
    $L_u=l_u(\tree)$ for each $u\in \tree\setminus(\spine\cup\spine')$ with $\abs{u}<n$,
\item
    %$(\dot L_{(v_1..v_m)},\dot C_{(v_1..v_m)})=(l_{(v_1..v_m)}(\tree),v_{m+1})$ for each $(v_1..v_mv_{m+1})\in\spine\cup\spine'$
    $(\dot L_{(v_0\cdots v_m)},\dot C_{(v_0\cdots v_m)})=(l_{(v_0\cdots v_m)}(\tree),v_{m+1})$ for each $(v_0\cdots v_mv_{m+1})\in\spine\cup\spine'$
    with $k_n\neq m<n$ and
\item
    %$(\ddot L,\ddot C,\ddot C')=(l_{(v_1..v_{k_n})}(\tree),v_{k_n+1},v'_{k_n+1})$ for $(v_1..v_{k_n}v_{k_n+1})\in\spine$ and $(v_1..v_{k_n}v'_{k_n+1})\in\spine'$.
    $(\ddot L,\ddot C,\ddot C')=(l_{(v_1\cdots v_{k_n})}(\tree),v_{k_n+1},v'_{k_n+1})$ for $(v_1\cdots v_{k_n}v_{k_n+1})\in\spine$ and $(v_1\cdots v_{k_n}v'_{k_n+1})\in\spine'$.
\end{itemize}
    Following this analysis, one can verify that
\begin{multline*}
		\prob\set{(\ddot T,\ddot V,\ddot V')=(\tree,\spine,\spine')}
	=
		\frac{1}{n}
	\cdot
	    \prod_{u\in \tree\setminus(\spine\cup \spine'):|u|<n}\mu\set{l_u(\tree)}
	\cdot
	    \prod_{u\in \spine\cup \spine':k_n\neq|u|<n}\frac{l_u(\tree)\mu\set{l_u(\tree)}}{m}\frac{1}{l_u(\tree)}
    \\\cdot
		\prod_{u\in \spine\cup \spine':|u|=k_n}\frac{l_u(\tree)(l_u(\tree)-1)\mu\set{l_u(\tree)}}{\sigma^2}\frac{1}{l_u(\tree)(l_u(\tree)-1)}
	=
		\frac{1}{n\sigma^2}\bG\set{\tree}.
\end{multline*}
\par	
	The \defn{$k(k-1)$-type size-biased Galton-Watson measure $\ddot \bG_n$} on space $\bbT_{\leq n}$ is then defined as the law of the random element $\ddot T$. That is, for any $\tree\in\bbT_{\leq n}$,
\begin{equation}
\label{eq:k(k-1)typesizebiasedGWmeasure}
\begin{split}
		\ddot \bG_n\set{\tree}
	&:=
		\bP\set{\ddot T=\tree}
	=
		\sum_{(\spine,\spine'):(\tree,\spine,\spine')\in \ddot\bbT_n} \bP\set{(\ddot T,\ddot V,\ddot V')=(\tree,\spine,\spine')}
	\\&=
	    \#\set{(\spine,\spine'):(\tree,\spine,\spine')\in \ddot\bbT_n}
	\cdot
	    \frac{\bG_n\set{\tree}}{n\sigma^2}
	=
		\frac{X_n(\tree)(X_n(\tree)-1)}{n\sigma^2}
    \cdot
        \bG_n\set{\tree}.
\end{split}
\end{equation}

\medskip
\begin{proof}[Proof of Theorem \ref{thm:changeofmeasure}]
    Since the process
    %$(X_m(\tree))_{m=0..n}$
    $(X_m(\tree))_{m=0,\cdots,n}$
    under $\bG_n$ is exactly a Galton-Watson branching process stopped at generation $n$, we have that
    %$((X_m(\tree))_{m=0..n};\bG_n) \eqlaw (Z_m)_{m=0..n}.$
    $((X_m(\tree))_{m=0,\cdots,n};\bG_n) \eqlaw (Z_m)_{m=0,\cdots,n}.$
    Similarly, since the process $(X_m(\tree))_{m=0,\cdots,n}$ under measure $\ddot \bG_n$ is constructed as the population of a $k(k-1)$-type size-biased Galton-Watson tree with height $n$, we have that $((X_m(\tree))_{m=0,\cdots,n};\ddot \bG_n) \eqlaw (\ddot Z_m)_{m=0,\cdots,n}.$ Rewrite equation \eqref{eq:k(k-1)typesizebiasedGWmeasure} into its Radon-Nikodym derivative form,
\begin{equation*}
    \frac{\ddot \bG_n(d\tree)}{\bG_n(d\tree)}=\frac{X_n(\tree)(X_n(\tree)-1)}{n\sigma^2}
    %, (\tree\in\bbT_{\leq n}).
    , \quad\tree\in\bbT_{\leq n}.
\end{equation*}
    For any bounded Borel function $g$ on $\bbN_0^n$, we can then verify that
\begin{multline}
\label{eq:proofofchangeofmeasure}
        \expct\brac{g\expr{\ddot Z_1^{(n)},\cdots,\ddot Z_n^{(n)}}}
	=
        \ddot \bG_n\brac{g\expr{X_1(\tree),\cdots, X_n(\tree)}}
    \\=
        \bG_n\brac{\frac{X_n(\tree)(X_n(\tree)-1)}{n\sigma^2}g\expr{X_1(\tree),\cdots, X_n(\tree)}}
       =
		\frac{\expct\brac{Z_n(Z_n-1) g\expr{Z_1,\cdots,Z_n}}}{n\sigma^2}.
\end{multline}
%    Taking $g\equiv 1$ in equation \eqref{eq:proofofchangeofmeasure}, we get that $\expct[Z_n(Z_n-1)]=n\sigma^2$.
%\end{proof}
\end{proof}
Taking $g\equiv 1$ in equation \eqref{eq:proofofchangeofmeasure}, we get that $\expct[Z_n(Z_n-1)]=n\sigma^2$.

\subsection{Spines decomposition}
\label{sec:spinesdecomposition}
	For any particle $u=(u_1\cdots u_n)$, define
\begin{equation*}
	    \ancestor{u}
	:=
	    \set{(u_0u_1\cdots u_j):j=0,\cdots,n}
\end{equation*}
    as the \defn{descending family line from $\emptyset$ to $u$}. Consider the size-biased Galton-Watson tree $(\dot T,\dot V)$. Since $\|\dot V\|=n$, we must have, for any particle $u\in\cU$, $\|\ancestor{u}\cap\dot V\|\in\set{0,1,..,n}$. Therefore, the particles in $\dot T$ can be separated into $n+1$ parts,
\begin{equation*}
		\dot T
	=
		\bigcup_{k=0}^n\dot A_k
	:=
	    \bigcup_{k=0}^n\set{u\in\dot T:\norm{\ancestor{u}\cap \dot V}=k}.
\end{equation*}
	This seperation gives a natural decomposition of the $n$th population of $\dot T$,
\begin{equation}
\label{eq:generationseperation}
		X_n(\dot T)
	=
		\sum_{k=0}^nX_n(\dot A_k).
\end{equation}
	With a closer look, \eqref{eq:generationseperation} is exactly the spine decomposition \eqref{eq:rawdecomposition} in a sense that
\begin{equation*}
        \expr{X_n(\dot T),\expr{X_n(\dot A_k)}_{k=0,\cdots,n-1},X_n(\dot A_n)}
    \eqlaw
        \expr{\dot Z_n,\expr{\dot Z_n^{(k+1)}}_{k=0,\cdots,n-1},1}.
\end{equation*}
\par
	Similarly, consider the $k(k-1)$-type size-biased Galton-Watson tree $(\ddot T,\ddot V,\ddot V')$. Since $\|\ddot V\cup\ddot V'\|=n$, we must have, for any particle $u\in\cU$, $\|\ancestor{u}\cap(\ddot V\cup \ddot V')\|\in\{0,1,..,n\}$. Therefore, the particles in $\ddot T$ can also be separated into $n+1$ parts,
\begin{equation*}
		\ddot T
	=
		\bigcup_{k=0}^n \ddot A_k
	:=
		\bigcup_{k=0}^n\set{u\in\ddot T:\norm{\ancestor{u}\cap\expr{\ddot V\cup\ddot V'}}=k}.
\end{equation*}
    This separation gives a natural decomposition of the $n$th population of $\ddot T$,
\begin{equation}
\label{eq:rawtwospinedecomposition}
		X_n(\ddot T)
	=
		\sum_{k=0}^nX_n(\ddot A_k)
	=
	    \sum_{k=0}^{K_n-1}X_n(\ddot A_k)
	+
		X_n(\ddot A_{K_n})
	+
		\sum_{k=K_n+1}^nX_n(\ddot A_k).
\end{equation}
	Taking $m\in\{0,\cdots,n-1\}$ and conditioning on $K_n=m$, we may argue from the construction of $(\ddot T,\ddot V,\ddot V')$ that:
\begin{itemize}
\item
    for any $k=0,\cdots,m-1$, $X_n(\ddot A_k)\eqlaw Z'_{n-k-1}$,
\item
    $X_n(\ddot A_m)\eqlaw Z''_{n-m-1}$,
\item
    $\sum_{k=m+1}^nX_n(\ddot A_k)\eqlaw \dot Z_{n-m-1}^{(1)}+\dot Z_{n-m-1}^{(2)}$ and
\item
    $\{X_n(\ddot A_k):k=0,\cdots,n\}$ are independent of each other.
\end{itemize}
    Here, $(Z_k')_{k\ge 0}$ and $(Z_k'')_{k\ge 0}$ are $\mu$-Galton-Watson branching processes with initial population distributed according to $\dot L-1$ and $\ddot L-2$ respectively. $(\dot Z_k^{(1)})_{k\ge 0}$ and $(\dot Z_k^{(2)})_{k\ge 0}$ are two independent copies of the size-biased $\mu$-Galton-Watson branching process $(\dot Z_k)_{k\ge 0}$. Following those arguments, taking Laplace transform, we may rewrite \eqref{eq:rawtwospinedecomposition} into,
\begin{equation}
\label{eq:laplacetransformationoftwospinedecomposition}
\begin{split}
		\cL_{\ddot Z_n^{(n)}}(\lambda)
	&=
		\prob\brac{e^{-\lambda X_n(\ddot T)}}
	=
	    \sum_{m=0}^{n-1}\prob\set{K_n=m}\prob\brac{e^{-\lambda X_n(\ddot T)}\bigg| K_n=m}
	\\&=
        \frac{1}{n}\sum_{m=0}^{n-1}\prod_{k=0}^{m-1}\cL_{Z'_{n-k-1}}(\lambda)
    \cdot
        \cL_{Z''_{n-m-1}}(\lambda)\cL_{\dot Z_{n-m-1}}^2(\lambda)
    %,(\lambda\geq 0).
    ,\quad\lambda\geq 0.
\end{split}
\end{equation}
\medskip
\begin{proof}[Proof of Proposition \ref{lem:twospinedecomposition}]
    It follows from \eqref{eq:spinedecomposition} that,
    %moved for below
    for any $\lambda\geq 0$,
\begin{multline}
\label{eq:keytakingback}
	    \frac{\cL_{\dot Z_n}(\lambda)}{\cL_{Z_{n-m-1}'}(\lambda)\cL_{\dot Z_{n-m-1}}(\lambda)}
	=
	    \cL_{Z_{n-m-1}'}^{-1}(\lambda)\frac{e^{-\lambda}\prod_{m=0}^{n-1}\cL_{Z'_m}(\lambda)}{e^{-\lambda}\prod_{m=0}^{n-m-2}\cL_{Z'_m}(\lambda)}
	=
	    \prod_{k=0}^{m-1}\cL_{Z'_{n-k-1}}(\lambda).
	%,(\lambda\geq 0).
%moved forward
\end{multline}
    Taking \eqref{eq:keytakingback} back into \eqref{eq:laplacetransformationoftwospinedecomposition}, we get that, for any $\lambda\geq 0$,
\begin{equation}
\label{eq:keyrelation}
        \cL_{\ddot Z_n^{(n)}}(\lambda)
	=
        \frac{1}{n}\sum_{m=0}^{n-1}\cL_{\dot Z_n}(\lambda)\frac{\cL_{Z''_{n-m-1}}(\lambda)}{\cL_{Z_{n-m-1}'}(\lambda)}\cL_{\dot Z_{n-m-1}}(\lambda)
    =
        \frac{1}{n}\cL_{\dot Z_n}(\lambda)\sum_{m=0}^{n-1}\frac{\cL_{Z''_m}(\lambda)}{\cL_{Z'_m}(\lambda)}\cL_{\dot Z_m}(\lambda).
\end{equation}
    According to \eqref{eq:htransformation} and Theorem \ref{thm:changeofmeasure}, $\dot Z_n$ and $\ddot Z_n^{(n)}$ have the size-biased and $k(k-1)$-type size-biased distribution of $Z_n$ respectively. Therefore
\begin{equation}
\label{eq:firstderivative}
        \cL_{\dot Z_n}(\lambda)
	=
	    \expct\brac{Z_n e^{-\lambda Z_n}}
	=
	    -\cL_{Z_n}'(\lambda)
	%,(\lambda > 0),
  ,\quad \lambda > 0,
\end{equation}
    and
\begin{equation}
\label{eq:secondderivative}
        \cL_{\ddot Z_n}(\lambda)
	=
		\frac{1}{n\sigma^2}\expct\brac{Z_n(Z_n-1)e^{-\lambda Z_n}}
	=
		\frac{1}{n\sigma^2}\expr{\cL_{Z_n}''(\lambda)+\cL_{Z_n}'(\lambda)}
	%,(\lambda > 0).
 ,\quad \lambda > 0.
\end{equation}
It then follows from \eqref{eq:keyrelation}, \eqref{eq:firstderivative} and \eqref{eq:secondderivative} that
\begin{equation*}
		-\frac{d}{d\lambda}\ln \cL_{\dot Z_n}(\lambda)
	=
		-\frac{\cL_{Z_n}''(\lambda)}{\cL_{Z_n}'(\lambda)}
	=
		1+\sigma^2\sum_{m=0}^{n-1}\frac{\cL_{Z''_{m}}(\lambda)}{\cL_{Z'_{m}}(\lambda)}\cL_{\dot Z_{m}}(\lambda),
	%(\lambda > 0).
\quad \lambda > 0.
\end{equation*}
	%Integration gives that,
Integrating from $0$ to $\lambda$ gives that,
for any $\lambda\geq 0$,
\begin{equation*}
		-\ln \cL_{\dot Z_n}(\lambda)
	=
		\lambda
	+
		\sigma^2\sum_{m=0}^{n-1}\int_0^\lambda \frac{\cL_{Z''_{m}}(s)}{\cL_{Z'_{m}}(s)}\cL_{Z_m}(s)ds.
\end{equation*}
\end{proof}
\section{A new proof of Yaglom's law}
\label{sec:anewproofofyaglomslaw}
\begin{lem}
\label{lem:zeroinequality}
    %Suppose that $c>0$ is constant,
    Suppose that $c>0$ is a constant,
    and function $F:[0,\infty)\to [0,1]$ satisfies that, for any $\lambda\geq 0$,
\begin{equation}
\label{eq:zeroinequality}
	    F(\lambda)
	\leq
	    \frac{1}{c}\int_0^1du
	\cdot
	    \int_0^\lambda F(us)ds.
\end{equation}
    Then $F\equiv 0$.
\end{lem}
\begin{proof}
   % We firstly
   We first
    show that $F(\lambda)=0$ while $\lambda \in [0,c)$.
    %Actually, taking $F(us)\leq 1$
    Using  $F(us)\leq 1$
    in the right hand of \eqref{eq:zeroinequality} gives us
\begin{equation*}
        F(\lambda)
    \leq
        \frac{1}{c}\int_0^1du
    \cdot
	    \int_0^\lambda ds
	=
	    \frac{\lambda}{c}
	%, (\lambda\geq 0).
,\quad \lambda\geq 0.
\end{equation*}
    Taking this new upper bound back into the right hand of \eqref{eq:zeroinequality} gives an updated upper bound of $F$
    %,
    :
\begin{equation*}
        F(\lambda)
    \leq
        \frac{1}{c}\int_0^1du
    \cdot
	    \int_0^\lambda \frac{us}{c}ds
	\leq
        \frac{1}{c}\int_0^1du
    \cdot
	    \int_0^\lambda \frac{\lambda}{c}ds
	=
	    \expr{\frac{\lambda}{c}}^2
	%, (\lambda\geq 0).
,\quad \lambda\geq 0.
\end{equation*}
    Repeating this process, we will have $F(\lambda)\leq (\frac{\lambda}{c})^m$ for any $m>0$. Therefore $F=0$ on $[0,c)$.
\par
    To complete the proof, we then show that, for any $k\in\bbN$, $F=0$ on $[0,kc)$ implies that $F=0$ on $[0,(k+1)c)$. Actually, while $F=0$ on $[0,kc)$, we have
\begin{equation*}
	    F\expr{kc+\lambda}
	\leq
	    c^{-1}\int_0^1 du\cdot\int_0^{kc+\lambda}F(us)ds
	=
	    c^{-1}\int_0^1du\cdot\int_{kc}^{kc+\lambda} F(us)ds\leq\frac{\lambda}{c}
	%,(\lambda\geq 0).
,\quad \lambda\geq 0.
\end{equation*}
	Using this technique again, we get that
\begin{equation*}
	    F(kc+\lambda)
	\leq
	    c^{-1}\int_0^1du\cdot\int_{kc}^{kc+\lambda} F(us)ds
	\leq
	    c^{-1}\int_0^1du\cdot\int_{kc}^{kc+\lambda} \frac{\lambda}{c}ds
	\leq\expr{\frac{\lambda}{c}}^2
   % ,(\lambda\geq 0).
   ,\quad \lambda\geq 0.
\end{equation*}
	Repeating this process gives that $F(kc+\lambda)\leq (\frac{\lambda}{c})^m$ for any $m>0$. Therefore $F=0$ on $[0,(k+1)c)$. The rest of the proof follows by induction.
\end{proof}

\medskip
%\begin{proof}[Proof of \eqref{thm:yaglom} of Theorem \ref{thm:kesten}.]
\begin{proof}[Proof of   Theorem \ref{thm:kesten} \eqref{thm:yaglom}.]
	Write $\nu:=\frac{2}{\sigma^2}$. Define
\begin{equation}
		M(a,\lambda)
	:=
		\abs{\cL_{\frac{\dot Z_{[a]}}{a}}(\lambda)-\frac{\nu^2}{(\nu+\lambda)^2}}
	%,(\lambda \geq 0).
,\quad\lambda \geq 0.
\end{equation}
	We firstly want to show that, for any $\lambda\geq 0$,
\begin{equation}
\label{eq:Miszerofunction}
		M(\lambda)
	:=
		\limsup_{a\to\infty}M(a,\lambda)
	=
		0.
\end{equation}
	Actually, replacing $\lambda$ with $\frac{\lambda}{n}$ in \eqref{eq:twospinedecomposition}, we have that, for any $\lambda\geq 0$,
\begin{equation}
\label{eq:passingtolimitequation}
\begin{split}
		-\ln \cL_{\frac{\dot Z_n}{n}}(\lambda)
	&=
		\frac{\lambda}{n}+\sigma^2\sum_{m=0}^{n-1}\int_0^{\frac{\lambda}{n}} \frac{\cL_{Z''_m}(s)}{\cL_{Z'_m}(s)}\cL_{\dot Z_{m}}(s)ds
	\\&=
		\frac{\lambda}{n}+\frac{2}{\nu}\int_0^1 du \cdot \int_0^{\lambda} \frac{\cL_{Z''_{[un]}}\expr{\frac{s}{n}}}{\cL_{Z'_{[un]}}\expr{\frac{s}{n}}}\cL_{\frac{\dot Z_{[un]}}{un}}(us)ds.
\end{split}
\end{equation}
	On the other hand, it's elementary calculus that
\begin{equation}
\label{eq:limitequation}
        -\ln\frac{\nu^2}{(\nu+\lambda)^2}
    =
		\frac{2}{\nu}\int_0^1du\cdot\int_0^\lambda \frac{\nu^2}{(\nu+us)^2}ds,
	%(\lambda\geq 0).
\quad \lambda\geq 0.
\end{equation}
	%Following
Using
\eqref{eq:passingtolimitequation}, \eqref{eq:limitequation} and the obvious fact that $\abs{a-b}\leq\abs{\ln a-\ln b}$ for any $0<a,b\leq 1$, we conclude that, for any $\lambda\geq 0$,
\begin{equation}
\label{eq:inequalitybeforepassingtolimit}
\begin{split}
        M(n,\lambda)
    &\leq
		\abs{\ln \cL_{\frac{\dot Z_n}{n}}(\lambda)-\ln\frac{\nu^2}{(\nu+\lambda)^2}}
	\\&\leq
		\frac{\lambda}{n}
	+
	    \frac{2}{\nu}\int_0^1 du
	\cdot
	    \int_0^{\lambda} \abs{\frac{\cL_{Z''_{[un]}}\expr{\frac{s}{n}}}{\cL_{Z'_{[un]}}\expr{\frac{s}{n}}}\cL_{\frac{\dot Z_{[un]}}{un}}(us)- \frac{\nu^2}{(\nu+us)^2}}ds.
\end{split}
\end{equation}
    %Passing $n$ to $\infty$ in
    Taking $\limsup_{n\to\infty}$ in
    \eqref{eq:inequalitybeforepassingtolimit}, using \eqref{eq:uniformly} and the Reverse Fatou's Lemma, we arrive at
\begin{equation*}
	    M(\lambda)
    \leq
        \frac{2}{\nu}\int_0^1du
    \cdot
        \int_0^\lambda M(us)ds
    ,(\lambda\geq 0).
\end{equation*}
	Consequently, according to Lemma \ref{lem:zeroinequality}, $M\equiv 0$.
\par
	The rest of the proof follows a standard tightness argument. For the Galton-Watson branching process $((Z_n)_{n\ge 0};\prob)$, write $\prob_n^*[\cdot]:=\prob[\cdot|Z_n>0]$. By Kolmogorov's estimate, \eqref{thm:kolmogorov} of Theorem \ref{thm:kesten}, 
%one can calculate that
we have
\begin{equation*}
	    \expct_n^*\brac{\frac{Z_n}{n}}
	=
	    \expct\brac{\frac{1_{Z_n>0}}{\prob\set{Z_n>0}}\frac{Z_n}{n}}
	=
	    \frac{\expct\brac{Z_n}}{n\prob\set{Z_n>0}}
    \to
        \frac{1}{\nu}
   % ,(n\to\infty).
   \quad\mbox{as }n\to\infty.
\end{equation*}
	As a consequence, laws of $(\frac{Z_n}{n};\prob_n^*)_{n\ge 1}$ are tight and there will be some sub-sequence, say $(\frac{Z_{n_k}}{n_k};\prob_{n_k}^*)_{k\ge 1}$, converging in law to some non-negative random variable, say $Y$. For any fixed $\lambda >0$, from the fact that function $x\mapsto 1_{x\geq 0}xe^{-\lambda x}$ is bounded and continuous, we have
\begin{equation}
\label{eq:subsequence}
	    \expct_{n_k}^*\brac{\frac{Z_{n_k}}{n_k}\exp\set{-\lambda\frac{Z_{n_k}}{n_k}}}
	\to
	    \expct\brac{Ye^{-\lambda Y}},
	%(k\to\infty).
\quad\mbox{as }k\to\infty.
\end{equation}
	On the other hand, using \eqref{eq:htransformation}, 
%we can verify that, 
we see that,
for each $n\in\bbN$,
\begin{equation}
\label{eq:sizebiasedandcondition}
	    \expct\brac{e^{-\lambda\frac{\dot Z_n}{n}}}
	=
	    \expct\brac{Z_n e^{-\lambda\frac{Z_n}{n}}}
	=
	    n\prob\set{Z_n>0}\expct_n^*\brac{\frac{Z_n}{n}e^{-\lambda\frac{Z_n}{n}}}.
\end{equation}
	Taking $n=n_k$ and
%passing $k\to\infty$,
letting $k\to\infty$,
it follows from Kolmogorov's estimate, \eqref{eq:subsequence} and \eqref{eq:sizebiasedandcondition} that
\begin{equation*}
	    \expct\brac{e^{-\lambda\frac{\dot Z_{n_k}}{n_k}}}
	\to
	    \nu\expct\brac{Ye^{-\lambda Y}}.
\end{equation*}
%	Comparing this with \eqref{eq:Miszerofunction}, we must have,
Then by \eqref{eq:firstderivative} and \eqref{eq:Miszerofunction}, we have,
 for any $\lambda >0$,
\begin{equation*}
	    -\nu\cL_Y'(\lambda)
	=
	    \nu\expct\brac{Ye^{-\lambda Y}}
    =	
	    \frac{\nu^2}{(\nu+\lambda)^2}
%.
,
\end{equation*}
%new added
which says $\cL_Y'(\lambda)=-\frac{\nu}{(\nu+\lambda)^2}$.
%end new
	Integration then gives that
\begin{equation*}
	    \cL_Y(\lambda)
	=
	    1
	+
	    \int_0^\lambda\frac{-\nu}{(\nu+s)^2}ds
	=
	    \frac{\nu}{\nu+\lambda}.
\end{equation*}
	So, $Y$ must be an exponentially distributed random variable with mean $\nu^{-1}=\frac{\sigma^2}{2}$. In particular, $Y$'s distribution is independent of the choice of $n_k$, and hence we have the convergence in law for the whole sequence, as desired.
\end{proof}
\bibliographystyle{plainnat}
\bibliography{YaglomGW}
\end{document}
